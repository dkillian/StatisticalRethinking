% Options for packages loaded elsewhere
\PassOptionsToPackage{unicode}{hyperref}
\PassOptionsToPackage{hyphens}{url}
\PassOptionsToPackage{dvipsnames,svgnames,x11names}{xcolor}
%
\documentclass[
  letterpaper,
  DIV=11,
  numbers=noendperiod,
  oneside]{scrartcl}

\usepackage{amsmath,amssymb}
\usepackage{iftex}
\ifPDFTeX
  \usepackage[T1]{fontenc}
  \usepackage[utf8]{inputenc}
  \usepackage{textcomp} % provide euro and other symbols
\else % if luatex or xetex
  \usepackage{unicode-math}
  \defaultfontfeatures{Scale=MatchLowercase}
  \defaultfontfeatures[\rmfamily]{Ligatures=TeX,Scale=1}
\fi
\usepackage{lmodern}
\ifPDFTeX\else  
    % xetex/luatex font selection
\fi
% Use upquote if available, for straight quotes in verbatim environments
\IfFileExists{upquote.sty}{\usepackage{upquote}}{}
\IfFileExists{microtype.sty}{% use microtype if available
  \usepackage[]{microtype}
  \UseMicrotypeSet[protrusion]{basicmath} % disable protrusion for tt fonts
}{}
\makeatletter
\@ifundefined{KOMAClassName}{% if non-KOMA class
  \IfFileExists{parskip.sty}{%
    \usepackage{parskip}
  }{% else
    \setlength{\parindent}{0pt}
    \setlength{\parskip}{6pt plus 2pt minus 1pt}}
}{% if KOMA class
  \KOMAoptions{parskip=half}}
\makeatother
\usepackage{xcolor}
\usepackage[left=1in,marginparwidth=2.0666666666667in,textwidth=4.1333333333333in,marginparsep=0.3in]{geometry}
\setlength{\emergencystretch}{3em} % prevent overfull lines
\setcounter{secnumdepth}{-\maxdimen} % remove section numbering
% Make \paragraph and \subparagraph free-standing
\ifx\paragraph\undefined\else
  \let\oldparagraph\paragraph
  \renewcommand{\paragraph}[1]{\oldparagraph{#1}\mbox{}}
\fi
\ifx\subparagraph\undefined\else
  \let\oldsubparagraph\subparagraph
  \renewcommand{\subparagraph}[1]{\oldsubparagraph{#1}\mbox{}}
\fi

\usepackage{color}
\usepackage{fancyvrb}
\newcommand{\VerbBar}{|}
\newcommand{\VERB}{\Verb[commandchars=\\\{\}]}
\DefineVerbatimEnvironment{Highlighting}{Verbatim}{commandchars=\\\{\}}
% Add ',fontsize=\small' for more characters per line
\usepackage{framed}
\definecolor{shadecolor}{RGB}{241,243,245}
\newenvironment{Shaded}{\begin{snugshade}}{\end{snugshade}}
\newcommand{\AlertTok}[1]{\textcolor[rgb]{0.68,0.00,0.00}{#1}}
\newcommand{\AnnotationTok}[1]{\textcolor[rgb]{0.37,0.37,0.37}{#1}}
\newcommand{\AttributeTok}[1]{\textcolor[rgb]{0.40,0.45,0.13}{#1}}
\newcommand{\BaseNTok}[1]{\textcolor[rgb]{0.68,0.00,0.00}{#1}}
\newcommand{\BuiltInTok}[1]{\textcolor[rgb]{0.00,0.23,0.31}{#1}}
\newcommand{\CharTok}[1]{\textcolor[rgb]{0.13,0.47,0.30}{#1}}
\newcommand{\CommentTok}[1]{\textcolor[rgb]{0.37,0.37,0.37}{#1}}
\newcommand{\CommentVarTok}[1]{\textcolor[rgb]{0.37,0.37,0.37}{\textit{#1}}}
\newcommand{\ConstantTok}[1]{\textcolor[rgb]{0.56,0.35,0.01}{#1}}
\newcommand{\ControlFlowTok}[1]{\textcolor[rgb]{0.00,0.23,0.31}{#1}}
\newcommand{\DataTypeTok}[1]{\textcolor[rgb]{0.68,0.00,0.00}{#1}}
\newcommand{\DecValTok}[1]{\textcolor[rgb]{0.68,0.00,0.00}{#1}}
\newcommand{\DocumentationTok}[1]{\textcolor[rgb]{0.37,0.37,0.37}{\textit{#1}}}
\newcommand{\ErrorTok}[1]{\textcolor[rgb]{0.68,0.00,0.00}{#1}}
\newcommand{\ExtensionTok}[1]{\textcolor[rgb]{0.00,0.23,0.31}{#1}}
\newcommand{\FloatTok}[1]{\textcolor[rgb]{0.68,0.00,0.00}{#1}}
\newcommand{\FunctionTok}[1]{\textcolor[rgb]{0.28,0.35,0.67}{#1}}
\newcommand{\ImportTok}[1]{\textcolor[rgb]{0.00,0.46,0.62}{#1}}
\newcommand{\InformationTok}[1]{\textcolor[rgb]{0.37,0.37,0.37}{#1}}
\newcommand{\KeywordTok}[1]{\textcolor[rgb]{0.00,0.23,0.31}{#1}}
\newcommand{\NormalTok}[1]{\textcolor[rgb]{0.00,0.23,0.31}{#1}}
\newcommand{\OperatorTok}[1]{\textcolor[rgb]{0.37,0.37,0.37}{#1}}
\newcommand{\OtherTok}[1]{\textcolor[rgb]{0.00,0.23,0.31}{#1}}
\newcommand{\PreprocessorTok}[1]{\textcolor[rgb]{0.68,0.00,0.00}{#1}}
\newcommand{\RegionMarkerTok}[1]{\textcolor[rgb]{0.00,0.23,0.31}{#1}}
\newcommand{\SpecialCharTok}[1]{\textcolor[rgb]{0.37,0.37,0.37}{#1}}
\newcommand{\SpecialStringTok}[1]{\textcolor[rgb]{0.13,0.47,0.30}{#1}}
\newcommand{\StringTok}[1]{\textcolor[rgb]{0.13,0.47,0.30}{#1}}
\newcommand{\VariableTok}[1]{\textcolor[rgb]{0.07,0.07,0.07}{#1}}
\newcommand{\VerbatimStringTok}[1]{\textcolor[rgb]{0.13,0.47,0.30}{#1}}
\newcommand{\WarningTok}[1]{\textcolor[rgb]{0.37,0.37,0.37}{\textit{#1}}}

\providecommand{\tightlist}{%
  \setlength{\itemsep}{0pt}\setlength{\parskip}{0pt}}\usepackage{longtable,booktabs,array}
\usepackage{calc} % for calculating minipage widths
% Correct order of tables after \paragraph or \subparagraph
\usepackage{etoolbox}
\makeatletter
\patchcmd\longtable{\par}{\if@noskipsec\mbox{}\fi\par}{}{}
\makeatother
% Allow footnotes in longtable head/foot
\IfFileExists{footnotehyper.sty}{\usepackage{footnotehyper}}{\usepackage{footnote}}
\makesavenoteenv{longtable}
\usepackage{graphicx}
\makeatletter
\def\maxwidth{\ifdim\Gin@nat@width>\linewidth\linewidth\else\Gin@nat@width\fi}
\def\maxheight{\ifdim\Gin@nat@height>\textheight\textheight\else\Gin@nat@height\fi}
\makeatother
% Scale images if necessary, so that they will not overflow the page
% margins by default, and it is still possible to overwrite the defaults
% using explicit options in \includegraphics[width, height, ...]{}
\setkeys{Gin}{width=\maxwidth,height=\maxheight,keepaspectratio}
% Set default figure placement to htbp
\makeatletter
\def\fps@figure{htbp}
\makeatother

\usepackage{booktabs}
\usepackage{longtable}
\usepackage{array}
\usepackage{multirow}
\usepackage{wrapfig}
\usepackage{float}
\usepackage{colortbl}
\usepackage{pdflscape}
\usepackage{tabu}
\usepackage{threeparttable}
\usepackage{threeparttablex}
\usepackage[normalem]{ulem}
\usepackage{makecell}
\usepackage{xcolor}
\usepackage{fontspec}
\usepackage{multicol}
\usepackage{hhline}
\newlength\Oldarrayrulewidth
\newlength\Oldtabcolsep
\usepackage{hyperref}
\KOMAoption{captions}{tableheading}
\makeatletter
\makeatother
\makeatletter
\makeatother
\makeatletter
\@ifpackageloaded{caption}{}{\usepackage{caption}}
\AtBeginDocument{%
\ifdefined\contentsname
  \renewcommand*\contentsname{Table of contents}
\else
  \newcommand\contentsname{Table of contents}
\fi
\ifdefined\listfigurename
  \renewcommand*\listfigurename{List of Figures}
\else
  \newcommand\listfigurename{List of Figures}
\fi
\ifdefined\listtablename
  \renewcommand*\listtablename{List of Tables}
\else
  \newcommand\listtablename{List of Tables}
\fi
\ifdefined\figurename
  \renewcommand*\figurename{Figure}
\else
  \newcommand\figurename{Figure}
\fi
\ifdefined\tablename
  \renewcommand*\tablename{Table}
\else
  \newcommand\tablename{Table}
\fi
}
\@ifpackageloaded{float}{}{\usepackage{float}}
\floatstyle{ruled}
\@ifundefined{c@chapter}{\newfloat{codelisting}{h}{lop}}{\newfloat{codelisting}{h}{lop}[chapter]}
\floatname{codelisting}{Listing}
\newcommand*\listoflistings{\listof{codelisting}{List of Listings}}
\makeatother
\makeatletter
\@ifpackageloaded{caption}{}{\usepackage{caption}}
\@ifpackageloaded{subcaption}{}{\usepackage{subcaption}}
\makeatother
\makeatletter
\@ifpackageloaded{tcolorbox}{}{\usepackage[skins,breakable]{tcolorbox}}
\makeatother
\makeatletter
\@ifundefined{shadecolor}{\definecolor{shadecolor}{rgb}{.97, .97, .97}}
\makeatother
\makeatletter
\makeatother
\makeatletter
\@ifpackageloaded{sidenotes}{}{\usepackage{sidenotes}}
\@ifpackageloaded{marginnote}{}{\usepackage{marginnote}}
\makeatother
\makeatletter
\makeatother
\ifLuaTeX
  \usepackage{selnolig}  % disable illegal ligatures
\fi
\IfFileExists{bookmark.sty}{\usepackage{bookmark}}{\usepackage{hyperref}}
\IfFileExists{xurl.sty}{\usepackage{xurl}}{} % add URL line breaks if available
\urlstyle{same} % disable monospaced font for URLs
\hypersetup{
  pdftitle={Statistical Rethinking 2024},
  colorlinks=true,
  linkcolor={blue},
  filecolor={Maroon},
  citecolor={Blue},
  urlcolor={Blue},
  pdfcreator={LaTeX via pandoc}}

\title{Statistical Rethinking 2024}
\usepackage{etoolbox}
\makeatletter
\providecommand{\subtitle}[1]{% add subtitle to \maketitle
  \apptocmd{\@title}{\par {\large #1 \par}}{}{}
}
\makeatother
\subtitle{Week 1 lecture notes}
\author{}
\date{2024-01-05}

\begin{document}
\maketitle
\ifdefined\Shaded\renewenvironment{Shaded}{\begin{tcolorbox}[breakable, boxrule=0pt, interior hidden, borderline west={3pt}{0pt}{shadecolor}, enhanced, frame hidden, sharp corners]}{\end{tcolorbox}}\fi

\renewcommand*\contentsname{Table of contents}
{
\hypersetup{linkcolor=}
\setcounter{tocdepth}{4}
\tableofcontents
}
\hypertarget{difference-between-bayesian-and-frequentist}{%
\subsection{Difference between Bayesian and
frequentist}\label{difference-between-bayesian-and-frequentist}}

The frequentist approach to inference:

\(P(data|model)\)

The Bayesian approach to inference:

\(P(model|data)\)

What are each of these expressions saying?

How would you summarize the different approaches narratively?

What estimation approach is used both by the frequentists and the
Bayesians?

\hypertarget{bayes-rule}{%
\subsection{Bayes Rule}\label{bayes-rule}}

Let's derive the Bayes' Rule, drawing primarily from
\href{https://oscarbonilla.com/2009/05/visualizing-bayes-theorem/}{Oscar
Bonilla}.

Consider two overlapping events A and B occurring within a universe U.

\begin{Shaded}
\begin{Highlighting}[]
\FunctionTok{include\_graphics}\NormalTok{(}\StringTok{"venn{-}last.png"}\NormalTok{)}
\end{Highlighting}
\end{Shaded}

\begin{figure}[H]

{\centering \includegraphics[width=1.3in,height=\textheight]{venn-last.png}

}

\end{figure}

Let's agree that the probability of something is how often it occurs. So
in the Venn diagram above:

\(P(A)=\frac{A}{U}\)

\(P(AB)=\frac{AB}{U}\)

What if we shifted our question to ask, how much of A is in B? Our new
universe is now B, and we're interested in the overlap of A and B as a
proportion of B. In the language of probability, for ``how much of A is
in B'' we say ``what is the probability of A, given B?'' In notation, we
write that as \(A|B\). This gives us:

\(P(A|B)=\frac{P(AB)}{P(B)}\)

Let's solve for \(P(AB)\):

\(P(AB)=P(A|B)P(B)\)

Now let's ask, how much of B is in A? In probability language, what is
the probability of B, given A?

\(P(B|A)=\frac{P(AB)}{P(A)}\)

Solve for \(P(AB)\):

\(P(AB)=P(B|A)P(A)\)

The joint probability of A and B is the same whether you are asking
about the proportion of A in B, or the proportion of B in A. Now as a
simple algebraic trick, put the two identities together and solve for
\(A|B\):

\(P(A|B)P(B)=P(B|A)P(A)\)

\(P(A|B)=\frac{P(B|A)P(A)}{P(B)}\)

And that's it. That's Bayes' Rule.

\hypertarget{bayes-rule-as-an-analytical-tool}{%
\subsubsection{Bayes' Rule as an analytical
tool}\label{bayes-rule-as-an-analytical-tool}}

Now let's interpret Bayes' Rule in terms of a data analysis. We want to
estimate the probability of a hypothesized cause, given its observed
effect. The hypothesized cause is our model of what's happening. The
observed effect is the set of data that we have collected.

\(P(model|data)=\frac{P(data|model)P(model)}{P(data)}\)

To express this narratively, we might say that we have some observed
data and we want to estimate the probability of various causes of what
we observed. That's \(P(model|data)\). We look at our data and ask, how
often would we see such data, if one of our hypothesized causes were
true? That's \(P(data|model)\), and is referred to as the
\emph{likelihood} of a proposed model, given the observed data. The
likelihood of a proposed model given the observed data is scaled by how
often that hypothesized cause occurs (\(P(model)\)). Note that
\(P(model)\) incorporates information that is outside of our statistical
framework. We call \(P(model)\) our \emph{prior probability}. We
multiply the likelihood of a model (given the observed data) by the
overall incidence of that model (\(P(model)\), our prior).

We then normalize the numerator of the expression to a probability. That
normalizing constant is \(P(data)\) and doesn't really have a
substantive interpretation other than to say it's the average
probability of your data which has the effect of converting your output
into a probability.

Practice:

Consider the following table:

\begin{Shaded}
\begin{Highlighting}[]
\NormalTok{cp }\OtherTok{\textless{}{-}} \FunctionTok{data.frame}\NormalTok{(}\AttributeTok{x=}\FunctionTok{c}\NormalTok{(}\DecValTok{1000}\NormalTok{,}\DecValTok{1500}\NormalTok{,}\DecValTok{2000}\NormalTok{),}
                 \AttributeTok{y10=}\FunctionTok{c}\NormalTok{(.}\DecValTok{802}\NormalTok{,.}\DecValTok{687}\NormalTok{, .}\DecValTok{456}\NormalTok{),}
                 \AttributeTok{y20=}\FunctionTok{c}\NormalTok{(.}\DecValTok{103}\NormalTok{, .}\DecValTok{172}\NormalTok{, .}\DecValTok{312}\NormalTok{),}
                 \AttributeTok{y30=}\FunctionTok{c}\NormalTok{(.}\DecValTok{034}\NormalTok{,.}\DecValTok{071}\NormalTok{, .}\DecValTok{124}\NormalTok{),}
                 \AttributeTok{y40=}\FunctionTok{c}\NormalTok{(.}\DecValTok{06}\NormalTok{,.}\DecValTok{071}\NormalTok{,.}\DecValTok{108}\NormalTok{),}
                 \AttributeTok{p\_x=}\FunctionTok{c}\NormalTok{(.}\DecValTok{116}\NormalTok{, .}\DecValTok{099}\NormalTok{, .}\DecValTok{785}\NormalTok{))}

\FunctionTok{flextable}\NormalTok{(cp)}
\end{Highlighting}
\end{Shaded}

\global\setlength{\Oldarrayrulewidth}{\arrayrulewidth}

\global\setlength{\Oldtabcolsep}{\tabcolsep}

\setlength{\tabcolsep}{0pt}

\renewcommand*{\arraystretch}{1.5}



\providecommand{\ascline}[3]{\noalign{\global\arrayrulewidth #1}\arrayrulecolor[HTML]{#2}\cline{#3}}

\begin{longtable*}[c]{|p{0.75in}|p{0.75in}|p{0.75in}|p{0.75in}|p{0.75in}|p{0.75in}}



\ascline{1.5pt}{666666}{1-6}

\multicolumn{1}{>{\raggedleft}m{\dimexpr 0.75in+0\tabcolsep}}{\textcolor[HTML]{000000}{\fontsize{11}{11}\selectfont{x}}} & \multicolumn{1}{>{\raggedleft}m{\dimexpr 0.75in+0\tabcolsep}}{\textcolor[HTML]{000000}{\fontsize{11}{11}\selectfont{y10}}} & \multicolumn{1}{>{\raggedleft}m{\dimexpr 0.75in+0\tabcolsep}}{\textcolor[HTML]{000000}{\fontsize{11}{11}\selectfont{y20}}} & \multicolumn{1}{>{\raggedleft}m{\dimexpr 0.75in+0\tabcolsep}}{\textcolor[HTML]{000000}{\fontsize{11}{11}\selectfont{y30}}} & \multicolumn{1}{>{\raggedleft}m{\dimexpr 0.75in+0\tabcolsep}}{\textcolor[HTML]{000000}{\fontsize{11}{11}\selectfont{y40}}} & \multicolumn{1}{>{\raggedleft}m{\dimexpr 0.75in+0\tabcolsep}}{\textcolor[HTML]{000000}{\fontsize{11}{11}\selectfont{p\_x}}} \\

\ascline{1.5pt}{666666}{1-6}\endfirsthead 

\ascline{1.5pt}{666666}{1-6}

\multicolumn{1}{>{\raggedleft}m{\dimexpr 0.75in+0\tabcolsep}}{\textcolor[HTML]{000000}{\fontsize{11}{11}\selectfont{x}}} & \multicolumn{1}{>{\raggedleft}m{\dimexpr 0.75in+0\tabcolsep}}{\textcolor[HTML]{000000}{\fontsize{11}{11}\selectfont{y10}}} & \multicolumn{1}{>{\raggedleft}m{\dimexpr 0.75in+0\tabcolsep}}{\textcolor[HTML]{000000}{\fontsize{11}{11}\selectfont{y20}}} & \multicolumn{1}{>{\raggedleft}m{\dimexpr 0.75in+0\tabcolsep}}{\textcolor[HTML]{000000}{\fontsize{11}{11}\selectfont{y30}}} & \multicolumn{1}{>{\raggedleft}m{\dimexpr 0.75in+0\tabcolsep}}{\textcolor[HTML]{000000}{\fontsize{11}{11}\selectfont{y40}}} & \multicolumn{1}{>{\raggedleft}m{\dimexpr 0.75in+0\tabcolsep}}{\textcolor[HTML]{000000}{\fontsize{11}{11}\selectfont{p\_x}}} \\

\ascline{1.5pt}{666666}{1-6}\endhead



\multicolumn{1}{>{\raggedleft}m{\dimexpr 0.75in+0\tabcolsep}}{\textcolor[HTML]{000000}{\fontsize{11}{11}\selectfont{1,000}}} & \multicolumn{1}{>{\raggedleft}m{\dimexpr 0.75in+0\tabcolsep}}{\textcolor[HTML]{000000}{\fontsize{11}{11}\selectfont{0.802}}} & \multicolumn{1}{>{\raggedleft}m{\dimexpr 0.75in+0\tabcolsep}}{\textcolor[HTML]{000000}{\fontsize{11}{11}\selectfont{0.103}}} & \multicolumn{1}{>{\raggedleft}m{\dimexpr 0.75in+0\tabcolsep}}{\textcolor[HTML]{000000}{\fontsize{11}{11}\selectfont{0.034}}} & \multicolumn{1}{>{\raggedleft}m{\dimexpr 0.75in+0\tabcolsep}}{\textcolor[HTML]{000000}{\fontsize{11}{11}\selectfont{0.060}}} & \multicolumn{1}{>{\raggedleft}m{\dimexpr 0.75in+0\tabcolsep}}{\textcolor[HTML]{000000}{\fontsize{11}{11}\selectfont{0.116}}} \\





\multicolumn{1}{>{\raggedleft}m{\dimexpr 0.75in+0\tabcolsep}}{\textcolor[HTML]{000000}{\fontsize{11}{11}\selectfont{1,500}}} & \multicolumn{1}{>{\raggedleft}m{\dimexpr 0.75in+0\tabcolsep}}{\textcolor[HTML]{000000}{\fontsize{11}{11}\selectfont{0.687}}} & \multicolumn{1}{>{\raggedleft}m{\dimexpr 0.75in+0\tabcolsep}}{\textcolor[HTML]{000000}{\fontsize{11}{11}\selectfont{0.172}}} & \multicolumn{1}{>{\raggedleft}m{\dimexpr 0.75in+0\tabcolsep}}{\textcolor[HTML]{000000}{\fontsize{11}{11}\selectfont{0.071}}} & \multicolumn{1}{>{\raggedleft}m{\dimexpr 0.75in+0\tabcolsep}}{\textcolor[HTML]{000000}{\fontsize{11}{11}\selectfont{0.071}}} & \multicolumn{1}{>{\raggedleft}m{\dimexpr 0.75in+0\tabcolsep}}{\textcolor[HTML]{000000}{\fontsize{11}{11}\selectfont{0.099}}} \\





\multicolumn{1}{>{\raggedleft}m{\dimexpr 0.75in+0\tabcolsep}}{\textcolor[HTML]{000000}{\fontsize{11}{11}\selectfont{2,000}}} & \multicolumn{1}{>{\raggedleft}m{\dimexpr 0.75in+0\tabcolsep}}{\textcolor[HTML]{000000}{\fontsize{11}{11}\selectfont{0.456}}} & \multicolumn{1}{>{\raggedleft}m{\dimexpr 0.75in+0\tabcolsep}}{\textcolor[HTML]{000000}{\fontsize{11}{11}\selectfont{0.312}}} & \multicolumn{1}{>{\raggedleft}m{\dimexpr 0.75in+0\tabcolsep}}{\textcolor[HTML]{000000}{\fontsize{11}{11}\selectfont{0.124}}} & \multicolumn{1}{>{\raggedleft}m{\dimexpr 0.75in+0\tabcolsep}}{\textcolor[HTML]{000000}{\fontsize{11}{11}\selectfont{0.108}}} & \multicolumn{1}{>{\raggedleft}m{\dimexpr 0.75in+0\tabcolsep}}{\textcolor[HTML]{000000}{\fontsize{11}{11}\selectfont{0.785}}} \\

\ascline{1.5pt}{666666}{1-6}



\end{longtable*}



\arrayrulecolor[HTML]{000000}

\global\setlength{\arrayrulewidth}{\Oldarrayrulewidth}

\global\setlength{\tabcolsep}{\Oldtabcolsep}

\renewcommand*{\arraystretch}{1}

The variable X takes on the values 1000, 1500, 2000

The variable Y takes on the values 10, 20, 30, 40

Of all elements with x=1000, 80 percent have a y value of 10. In
probability notation:

\(P(y=10|x=1000)=80.2\%\)

Note that this is a specific instantiation of \(P(Y|X)\), where X and Y
take on specific values. What expression did we derive for \(P(Y|X)\)?

Units with x=1000 make up 11.6 percent of the sample.
\(P(x=1000)=11.6\%\)

Based on this information, what proportion of the total sample has an x
value 1000 and y value of 10?

In probability language, this is the joint probability of X and Y, which
is \(P(XY)\) or, more formally, \(P(X\cap Y)\). What expression did we
derive for \(P(XY)\)?

Using the expression for \(P(XY)\) and information in the table provided
above, do the following calculations:

\(P(x=1000 \cap y=40)\)

\(P(x=2000\cap y=20)\)

\(P(x=1500\cap y=30)\)

Now consider another table:

\begin{Shaded}
\begin{Highlighting}[]
\NormalTok{bp }\OtherTok{\textless{}{-}} \FunctionTok{data.frame}\NormalTok{(}\AttributeTok{x=}\FunctionTok{c}\NormalTok{(}\DecValTok{1000}\NormalTok{,}\DecValTok{1500}\NormalTok{,}\DecValTok{2000}\NormalTok{, }\StringTok{"p\_y"}\NormalTok{),}
                 \StringTok{"y:10"}\OtherTok{=}\FunctionTok{c}\NormalTok{(.}\DecValTok{093}\NormalTok{, .}\DecValTok{068}\NormalTok{, .}\DecValTok{358}\NormalTok{, .}\DecValTok{519}\NormalTok{),}
                 \StringTok{\textasciigrave{}}\AttributeTok{y:20}\StringTok{\textasciigrave{}}\OtherTok{=}\FunctionTok{c}\NormalTok{(.}\DecValTok{012}\NormalTok{, .}\DecValTok{017}\NormalTok{, .}\DecValTok{245}\NormalTok{, .}\DecValTok{274}\NormalTok{),}
                 \StringTok{\textasciigrave{}}\AttributeTok{y:30}\StringTok{\textasciigrave{}}\OtherTok{=}\FunctionTok{c}\NormalTok{(.}\DecValTok{004}\NormalTok{, .}\DecValTok{007}\NormalTok{, .}\DecValTok{097}\NormalTok{, .}\DecValTok{108}\NormalTok{),}
                 \StringTok{\textasciigrave{}}\AttributeTok{y:40}\StringTok{\textasciigrave{}}\OtherTok{=}\FunctionTok{c}\NormalTok{(.}\DecValTok{007}\NormalTok{, .}\DecValTok{007}\NormalTok{, .}\DecValTok{085}\NormalTok{, .}\DecValTok{099}\NormalTok{),}
                 \StringTok{\textasciigrave{}}\AttributeTok{p\_x}\StringTok{\textasciigrave{}}\OtherTok{=}\FunctionTok{c}\NormalTok{(.}\DecValTok{116}\NormalTok{, .}\DecValTok{099}\NormalTok{, .}\DecValTok{785}\NormalTok{, }\ConstantTok{NA}\NormalTok{))}
\NormalTok{bp }\SpecialCharTok{\%\textgreater{}\%}
  \FunctionTok{flextable}\NormalTok{()}
\end{Highlighting}
\end{Shaded}

\global\setlength{\Oldarrayrulewidth}{\arrayrulewidth}

\global\setlength{\Oldtabcolsep}{\tabcolsep}

\setlength{\tabcolsep}{0pt}

\renewcommand*{\arraystretch}{1.5}



\providecommand{\ascline}[3]{\noalign{\global\arrayrulewidth #1}\arrayrulecolor[HTML]{#2}\cline{#3}}

\begin{longtable*}[c]{|p{0.75in}|p{0.75in}|p{0.75in}|p{0.75in}|p{0.75in}|p{0.75in}}



\ascline{1.5pt}{666666}{1-6}

\multicolumn{1}{>{\raggedright}m{\dimexpr 0.75in+0\tabcolsep}}{\textcolor[HTML]{000000}{\fontsize{11}{11}\selectfont{x}}} & \multicolumn{1}{>{\raggedleft}m{\dimexpr 0.75in+0\tabcolsep}}{\textcolor[HTML]{000000}{\fontsize{11}{11}\selectfont{y.10}}} & \multicolumn{1}{>{\raggedleft}m{\dimexpr 0.75in+0\tabcolsep}}{\textcolor[HTML]{000000}{\fontsize{11}{11}\selectfont{y.20}}} & \multicolumn{1}{>{\raggedleft}m{\dimexpr 0.75in+0\tabcolsep}}{\textcolor[HTML]{000000}{\fontsize{11}{11}\selectfont{y.30}}} & \multicolumn{1}{>{\raggedleft}m{\dimexpr 0.75in+0\tabcolsep}}{\textcolor[HTML]{000000}{\fontsize{11}{11}\selectfont{y.40}}} & \multicolumn{1}{>{\raggedleft}m{\dimexpr 0.75in+0\tabcolsep}}{\textcolor[HTML]{000000}{\fontsize{11}{11}\selectfont{p\_x}}} \\

\ascline{1.5pt}{666666}{1-6}\endfirsthead 

\ascline{1.5pt}{666666}{1-6}

\multicolumn{1}{>{\raggedright}m{\dimexpr 0.75in+0\tabcolsep}}{\textcolor[HTML]{000000}{\fontsize{11}{11}\selectfont{x}}} & \multicolumn{1}{>{\raggedleft}m{\dimexpr 0.75in+0\tabcolsep}}{\textcolor[HTML]{000000}{\fontsize{11}{11}\selectfont{y.10}}} & \multicolumn{1}{>{\raggedleft}m{\dimexpr 0.75in+0\tabcolsep}}{\textcolor[HTML]{000000}{\fontsize{11}{11}\selectfont{y.20}}} & \multicolumn{1}{>{\raggedleft}m{\dimexpr 0.75in+0\tabcolsep}}{\textcolor[HTML]{000000}{\fontsize{11}{11}\selectfont{y.30}}} & \multicolumn{1}{>{\raggedleft}m{\dimexpr 0.75in+0\tabcolsep}}{\textcolor[HTML]{000000}{\fontsize{11}{11}\selectfont{y.40}}} & \multicolumn{1}{>{\raggedleft}m{\dimexpr 0.75in+0\tabcolsep}}{\textcolor[HTML]{000000}{\fontsize{11}{11}\selectfont{p\_x}}} \\

\ascline{1.5pt}{666666}{1-6}\endhead



\multicolumn{1}{>{\raggedright}m{\dimexpr 0.75in+0\tabcolsep}}{\textcolor[HTML]{000000}{\fontsize{11}{11}\selectfont{1000}}} & \multicolumn{1}{>{\raggedleft}m{\dimexpr 0.75in+0\tabcolsep}}{\textcolor[HTML]{000000}{\fontsize{11}{11}\selectfont{0.093}}} & \multicolumn{1}{>{\raggedleft}m{\dimexpr 0.75in+0\tabcolsep}}{\textcolor[HTML]{000000}{\fontsize{11}{11}\selectfont{0.012}}} & \multicolumn{1}{>{\raggedleft}m{\dimexpr 0.75in+0\tabcolsep}}{\textcolor[HTML]{000000}{\fontsize{11}{11}\selectfont{0.004}}} & \multicolumn{1}{>{\raggedleft}m{\dimexpr 0.75in+0\tabcolsep}}{\textcolor[HTML]{000000}{\fontsize{11}{11}\selectfont{0.007}}} & \multicolumn{1}{>{\raggedleft}m{\dimexpr 0.75in+0\tabcolsep}}{\textcolor[HTML]{000000}{\fontsize{11}{11}\selectfont{0.116}}} \\





\multicolumn{1}{>{\raggedright}m{\dimexpr 0.75in+0\tabcolsep}}{\textcolor[HTML]{000000}{\fontsize{11}{11}\selectfont{1500}}} & \multicolumn{1}{>{\raggedleft}m{\dimexpr 0.75in+0\tabcolsep}}{\textcolor[HTML]{000000}{\fontsize{11}{11}\selectfont{0.068}}} & \multicolumn{1}{>{\raggedleft}m{\dimexpr 0.75in+0\tabcolsep}}{\textcolor[HTML]{000000}{\fontsize{11}{11}\selectfont{0.017}}} & \multicolumn{1}{>{\raggedleft}m{\dimexpr 0.75in+0\tabcolsep}}{\textcolor[HTML]{000000}{\fontsize{11}{11}\selectfont{0.007}}} & \multicolumn{1}{>{\raggedleft}m{\dimexpr 0.75in+0\tabcolsep}}{\textcolor[HTML]{000000}{\fontsize{11}{11}\selectfont{0.007}}} & \multicolumn{1}{>{\raggedleft}m{\dimexpr 0.75in+0\tabcolsep}}{\textcolor[HTML]{000000}{\fontsize{11}{11}\selectfont{0.099}}} \\





\multicolumn{1}{>{\raggedright}m{\dimexpr 0.75in+0\tabcolsep}}{\textcolor[HTML]{000000}{\fontsize{11}{11}\selectfont{2000}}} & \multicolumn{1}{>{\raggedleft}m{\dimexpr 0.75in+0\tabcolsep}}{\textcolor[HTML]{000000}{\fontsize{11}{11}\selectfont{0.358}}} & \multicolumn{1}{>{\raggedleft}m{\dimexpr 0.75in+0\tabcolsep}}{\textcolor[HTML]{000000}{\fontsize{11}{11}\selectfont{0.245}}} & \multicolumn{1}{>{\raggedleft}m{\dimexpr 0.75in+0\tabcolsep}}{\textcolor[HTML]{000000}{\fontsize{11}{11}\selectfont{0.097}}} & \multicolumn{1}{>{\raggedleft}m{\dimexpr 0.75in+0\tabcolsep}}{\textcolor[HTML]{000000}{\fontsize{11}{11}\selectfont{0.085}}} & \multicolumn{1}{>{\raggedleft}m{\dimexpr 0.75in+0\tabcolsep}}{\textcolor[HTML]{000000}{\fontsize{11}{11}\selectfont{0.785}}} \\





\multicolumn{1}{>{\raggedright}m{\dimexpr 0.75in+0\tabcolsep}}{\textcolor[HTML]{000000}{\fontsize{11}{11}\selectfont{p\_y}}} & \multicolumn{1}{>{\raggedleft}m{\dimexpr 0.75in+0\tabcolsep}}{\textcolor[HTML]{000000}{\fontsize{11}{11}\selectfont{0.519}}} & \multicolumn{1}{>{\raggedleft}m{\dimexpr 0.75in+0\tabcolsep}}{\textcolor[HTML]{000000}{\fontsize{11}{11}\selectfont{0.274}}} & \multicolumn{1}{>{\raggedleft}m{\dimexpr 0.75in+0\tabcolsep}}{\textcolor[HTML]{000000}{\fontsize{11}{11}\selectfont{0.108}}} & \multicolumn{1}{>{\raggedleft}m{\dimexpr 0.75in+0\tabcolsep}}{\textcolor[HTML]{000000}{\fontsize{11}{11}\selectfont{0.099}}} & \multicolumn{1}{>{\raggedleft}m{\dimexpr 0.75in+0\tabcolsep}}{\textcolor[HTML]{000000}{\fontsize{11}{11}\selectfont{}}} \\

\ascline{1.5pt}{666666}{1-6}



\end{longtable*}



\arrayrulecolor[HTML]{000000}

\global\setlength{\arrayrulewidth}{\Oldarrayrulewidth}

\global\setlength{\tabcolsep}{\Oldtabcolsep}

\renewcommand*{\arraystretch}{1}

\begin{Shaded}
\begin{Highlighting}[]
\FunctionTok{sum}\NormalTok{(bp[}\DecValTok{1}\SpecialCharTok{:}\DecValTok{3}\NormalTok{, }\DecValTok{2}\SpecialCharTok{:}\DecValTok{5}\NormalTok{])}
\end{Highlighting}
\end{Shaded}

\begin{verbatim}
[1] 1
\end{verbatim}

\begin{Shaded}
\begin{Highlighting}[]
\FunctionTok{sum}\NormalTok{(bp[}\DecValTok{4}\NormalTok{,}\DecValTok{2}\SpecialCharTok{:}\DecValTok{5}\NormalTok{])}
\end{Highlighting}
\end{Shaded}

\begin{verbatim}
[1] 1
\end{verbatim}

\begin{Shaded}
\begin{Highlighting}[]
\FunctionTok{sum}\NormalTok{(bp[,}\DecValTok{6}\NormalTok{], }\AttributeTok{na.rm=}\NormalTok{T)}
\end{Highlighting}
\end{Shaded}

\begin{verbatim}
[1] 1
\end{verbatim}

The probability of x=1000 and y=20 is 1.2 percent.
\(P(y=20\cap x=1000)=1.2\%\)

Among units with x=1000, what proportion have y=20? As mentioned above,
this is a specific instantiation of \(P(Y|X)\)

Using the expression for \(P(Y|X)\) and information in the table above,
do the following calculations:

\(P(y=20|x=1000)\)

\(P(y=40|x=2000)\)

\(P(y=10|x=1500)\)

\hypertarget{inference-for-a-proportion}{%
\subsection{Inference for a
proportion}\label{inference-for-a-proportion}}

Consider a randomly assigned treatment to 20 subjects, and 20 subjects
without the treatment

The treatment is expected to reduce the probability of an event. The 40
subjects are observed for a given period of time, and the incidence of
events is observed for each group.

For any events that occur during this period, what is the probability
\emph{p} that the event occurred within the treatment group?

\emph{p} = probability of the event originating from the treatment group

\(H_0: p=50\%\) The event is equally likely to have occurred in the
treatment or control group, signifying no treatment effect

\(H_1: p \lt 50\%\) The event is less likely to have come from the
treatment group than the control group, signifying a treatment effect

For the period under observation, there were 20 events - 4 events in the
treatment group and 16 events in the control group.

How likely are these four events to have occurred within the treatment
group?

\hypertarget{frequentist}{%
\subsubsection{Frequentist}\label{frequentist}}

Background video for this example may be found
\href{https://www.coursera.org/learn/bayesian/lecture/fVGe0/inference-for-a-proportion-frequentist-approach}{here}

We may consider the twenty events as the number of trials, and the four
events in the treatment group as the number of `successes' of these 20
trials.

\begin{Shaded}
\begin{Highlighting}[]
\FunctionTok{dbinom}\NormalTok{(}\AttributeTok{x=}\DecValTok{0}\SpecialCharTok{:}\DecValTok{4}\NormalTok{, }\CommentTok{\# we consider the probabilities for all possibilities up to four }
       \AttributeTok{size=}\DecValTok{20}\NormalTok{, }
       \AttributeTok{prob=}\NormalTok{.}\DecValTok{5}\NormalTok{) }\SpecialCharTok{\%\textgreater{}\%}
  \FunctionTok{sum}\NormalTok{()}
\end{Highlighting}
\end{Shaded}

\begin{verbatim}
[1] 0.00591
\end{verbatim}

The probability of observing four or fewer pregnancies, under an
assumption of no treatment effect (the null hypothesis), is .0059.

The probability is less than 5\%. By convention, we reject the null
hypothesis and consider this `failure to reject' to be a piece of
evidence in support of the alternative hypothesis.

\hypertarget{bayesian}{%
\subsubsection{Bayesian}\label{bayesian}}

Background video for this example may be found
\href{https://www.coursera.org/learn/bayesian/lecture/cOdqz/inference-for-a-proportion-bayesian-approach}{here}

Recall Bayes' Rule:

\(P(model|data)=\frac{P(data|model)P(model)}{P(data)}\)

Recall that Bayes' Rule gives us the probability of a hypothesis (model)
given the observed data - that's \(P(model|data)\)

We refer to \(P(data|model)\) as the \emph{likelihood} of our data for a
given hypothesized model.

We refer to \(P(model)\) as our prior - what we think is the probability
of the model absent any observed data.

Finally, \(P(data)\) has no substantive interpretation, it's just a
normalizing constant

Under a Bayesian framework, the null hypothesis remains the same, but we
search the model space for the most likely value given the observed
data.

What would be a sensible range of plausible values? Let's say that the
range of possible probabilities would be 10\% to 90\%. So we'll set our
model space:

\begin{Shaded}
\begin{Highlighting}[]
\NormalTok{p }\OtherTok{\textless{}{-}} \FunctionTok{seq}\NormalTok{(}\AttributeTok{from=}\NormalTok{.}\DecValTok{1}\NormalTok{,}
         \AttributeTok{to=}\NormalTok{.}\DecValTok{9}\NormalTok{,}
         \AttributeTok{by=}\NormalTok{.}\DecValTok{1}\NormalTok{)}
\NormalTok{p}
\end{Highlighting}
\end{Shaded}

\begin{verbatim}
[1] 0.1 0.2 0.3 0.4 0.5 0.6 0.7 0.8 0.9
\end{verbatim}

\hypertarget{likelihood}{%
\paragraph{Likelihood}\label{likelihood}}

Now that we have our model space and observed data, we can calculate the
likelihoods. We ask how often our observed data occur for each of our
candidate models.

\begin{Shaded}
\begin{Highlighting}[]
\FunctionTok{options}\NormalTok{(}\AttributeTok{digits=}\DecValTok{3}\NormalTok{, }\AttributeTok{scipen=}\DecValTok{10}\NormalTok{)}

\NormalTok{likelihood }\OtherTok{\textless{}{-}} \FunctionTok{dbinom}\NormalTok{(}\AttributeTok{x=}\DecValTok{4}\NormalTok{,}
                     \AttributeTok{size=}\DecValTok{20}\NormalTok{,}
                     \AttributeTok{prob=}\NormalTok{p)}
\NormalTok{likelihood }\SpecialCharTok{\%\textgreater{}\%}
  \FunctionTok{round}\NormalTok{(}\DecValTok{4}\NormalTok{)}
\end{Highlighting}
\end{Shaded}

\begin{verbatim}
[1] 0.0898 0.2182 0.1304 0.0350 0.0046 0.0003 0.0000 0.0000 0.0000
\end{verbatim}

\hypertarget{priors}{%
\paragraph{Priors}\label{priors}}

After we set our model space, we need to establish our priors

For each of the possible models, which ones do we think are more likely
to be true?

Well, we don't know if the treatment works, so we should probably put
the most weight on our null hypothesis of no effect. In this specific
setup, a null hypothesis of no treatment effect is 50\% - the event
could just as easily have occurred in the treatment group as in the
control group.

Now let's suppose that there may be some small chance that the treatment
has the desired effect. Let's also suppose that there may be some small
chance that the treatment has the opposite effect of what it is intended
to do. We might think a positive treatment effect is more likely than a
negative treatment effect. But to be fully agnostic and honest about our
experiment, we will weight both scenarios equally.

Putting most weight on the null hypothesis of 50\% and small weight
equally distributed on the possibility of a positive or negative
treatment effect, might look like this:

\begin{Shaded}
\begin{Highlighting}[]
\NormalTok{prior }\OtherTok{\textless{}{-}} \FunctionTok{c}\NormalTok{(}\FunctionTok{rep}\NormalTok{(.}\DecValTok{06}\NormalTok{,}\DecValTok{4}\NormalTok{), .}\DecValTok{52}\NormalTok{, }\FunctionTok{rep}\NormalTok{(.}\DecValTok{06}\NormalTok{,}\DecValTok{4}\NormalTok{))}
\NormalTok{prior}
\end{Highlighting}
\end{Shaded}

\begin{verbatim}
[1] 0.06 0.06 0.06 0.06 0.52 0.06 0.06 0.06 0.06
\end{verbatim}

Narratively, we might express our priors as saying that we are inclined
to believe that there is no treatment effect, but we are willing to
entertain the possibility that the treatment may have a positive or
negative effect.

We are now ready to calculate the numerator in Bayes' Rule. We simply
multiply the likelihood by its corresponding prior.

\begin{Shaded}
\begin{Highlighting}[]
\NormalTok{numerator }\OtherTok{\textless{}{-}}\NormalTok{ likelihood}\SpecialCharTok{*}\NormalTok{prior}

\NormalTok{numerator }\SpecialCharTok{\%\textgreater{}\%}
  \FunctionTok{round}\NormalTok{(}\DecValTok{3}\NormalTok{)}
\end{Highlighting}
\end{Shaded}

\begin{verbatim}
[1] 0.005 0.013 0.008 0.002 0.002 0.000 0.000 0.000 0.000
\end{verbatim}

The last step is to normalize our numerator values into a set of
probabilities.

\begin{Shaded}
\begin{Highlighting}[]
\NormalTok{denominator }\OtherTok{\textless{}{-}} \FunctionTok{sum}\NormalTok{(numerator)}

\NormalTok{posterior }\OtherTok{\textless{}{-}}\NormalTok{ numerator }\SpecialCharTok{/}\NormalTok{ denominator}

\NormalTok{posterior }\SpecialCharTok{\%\textgreater{}\%}
  \FunctionTok{round}\NormalTok{(}\DecValTok{4}\NormalTok{)}
\end{Highlighting}
\end{Shaded}

\begin{verbatim}
[1] 0.1748 0.4248 0.2539 0.0681 0.0780 0.0005 0.0000 0.0000 0.0000
\end{verbatim}

Let's put all this together in a summary table.

\begin{Shaded}
\begin{Highlighting}[]
\NormalTok{out }\OtherTok{\textless{}{-}} \FunctionTok{data.frame}\NormalTok{(p, likelihood, prior, numerator, posterior) }\SpecialCharTok{\%\textgreater{}\%}
  \FunctionTok{round}\NormalTok{(}\DecValTok{3}\NormalTok{)}

\FunctionTok{flextable}\NormalTok{(out)}
\end{Highlighting}
\end{Shaded}

\global\setlength{\Oldarrayrulewidth}{\arrayrulewidth}

\global\setlength{\Oldtabcolsep}{\tabcolsep}

\setlength{\tabcolsep}{0pt}

\renewcommand*{\arraystretch}{1.5}



\providecommand{\ascline}[3]{\noalign{\global\arrayrulewidth #1}\arrayrulecolor[HTML]{#2}\cline{#3}}

\begin{longtable*}[c]{|p{0.75in}|p{0.75in}|p{0.75in}|p{0.75in}|p{0.75in}}



\ascline{1.5pt}{666666}{1-5}

\multicolumn{1}{>{\raggedleft}m{\dimexpr 0.75in+0\tabcolsep}}{\textcolor[HTML]{000000}{\fontsize{11}{11}\selectfont{p}}} & \multicolumn{1}{>{\raggedleft}m{\dimexpr 0.75in+0\tabcolsep}}{\textcolor[HTML]{000000}{\fontsize{11}{11}\selectfont{likelihood}}} & \multicolumn{1}{>{\raggedleft}m{\dimexpr 0.75in+0\tabcolsep}}{\textcolor[HTML]{000000}{\fontsize{11}{11}\selectfont{prior}}} & \multicolumn{1}{>{\raggedleft}m{\dimexpr 0.75in+0\tabcolsep}}{\textcolor[HTML]{000000}{\fontsize{11}{11}\selectfont{numerator}}} & \multicolumn{1}{>{\raggedleft}m{\dimexpr 0.75in+0\tabcolsep}}{\textcolor[HTML]{000000}{\fontsize{11}{11}\selectfont{posterior}}} \\

\ascline{1.5pt}{666666}{1-5}\endfirsthead 

\ascline{1.5pt}{666666}{1-5}

\multicolumn{1}{>{\raggedleft}m{\dimexpr 0.75in+0\tabcolsep}}{\textcolor[HTML]{000000}{\fontsize{11}{11}\selectfont{p}}} & \multicolumn{1}{>{\raggedleft}m{\dimexpr 0.75in+0\tabcolsep}}{\textcolor[HTML]{000000}{\fontsize{11}{11}\selectfont{likelihood}}} & \multicolumn{1}{>{\raggedleft}m{\dimexpr 0.75in+0\tabcolsep}}{\textcolor[HTML]{000000}{\fontsize{11}{11}\selectfont{prior}}} & \multicolumn{1}{>{\raggedleft}m{\dimexpr 0.75in+0\tabcolsep}}{\textcolor[HTML]{000000}{\fontsize{11}{11}\selectfont{numerator}}} & \multicolumn{1}{>{\raggedleft}m{\dimexpr 0.75in+0\tabcolsep}}{\textcolor[HTML]{000000}{\fontsize{11}{11}\selectfont{posterior}}} \\

\ascline{1.5pt}{666666}{1-5}\endhead



\multicolumn{1}{>{\raggedleft}m{\dimexpr 0.75in+0\tabcolsep}}{\textcolor[HTML]{000000}{\fontsize{11}{11}\selectfont{0.1}}} & \multicolumn{1}{>{\raggedleft}m{\dimexpr 0.75in+0\tabcolsep}}{\textcolor[HTML]{000000}{\fontsize{11}{11}\selectfont{0.090}}} & \multicolumn{1}{>{\raggedleft}m{\dimexpr 0.75in+0\tabcolsep}}{\textcolor[HTML]{000000}{\fontsize{11}{11}\selectfont{0.06}}} & \multicolumn{1}{>{\raggedleft}m{\dimexpr 0.75in+0\tabcolsep}}{\textcolor[HTML]{000000}{\fontsize{11}{11}\selectfont{0.005}}} & \multicolumn{1}{>{\raggedleft}m{\dimexpr 0.75in+0\tabcolsep}}{\textcolor[HTML]{000000}{\fontsize{11}{11}\selectfont{0.175}}} \\





\multicolumn{1}{>{\raggedleft}m{\dimexpr 0.75in+0\tabcolsep}}{\textcolor[HTML]{000000}{\fontsize{11}{11}\selectfont{0.2}}} & \multicolumn{1}{>{\raggedleft}m{\dimexpr 0.75in+0\tabcolsep}}{\textcolor[HTML]{000000}{\fontsize{11}{11}\selectfont{0.218}}} & \multicolumn{1}{>{\raggedleft}m{\dimexpr 0.75in+0\tabcolsep}}{\textcolor[HTML]{000000}{\fontsize{11}{11}\selectfont{0.06}}} & \multicolumn{1}{>{\raggedleft}m{\dimexpr 0.75in+0\tabcolsep}}{\textcolor[HTML]{000000}{\fontsize{11}{11}\selectfont{0.013}}} & \multicolumn{1}{>{\raggedleft}m{\dimexpr 0.75in+0\tabcolsep}}{\textcolor[HTML]{000000}{\fontsize{11}{11}\selectfont{0.425}}} \\





\multicolumn{1}{>{\raggedleft}m{\dimexpr 0.75in+0\tabcolsep}}{\textcolor[HTML]{000000}{\fontsize{11}{11}\selectfont{0.3}}} & \multicolumn{1}{>{\raggedleft}m{\dimexpr 0.75in+0\tabcolsep}}{\textcolor[HTML]{000000}{\fontsize{11}{11}\selectfont{0.130}}} & \multicolumn{1}{>{\raggedleft}m{\dimexpr 0.75in+0\tabcolsep}}{\textcolor[HTML]{000000}{\fontsize{11}{11}\selectfont{0.06}}} & \multicolumn{1}{>{\raggedleft}m{\dimexpr 0.75in+0\tabcolsep}}{\textcolor[HTML]{000000}{\fontsize{11}{11}\selectfont{0.008}}} & \multicolumn{1}{>{\raggedleft}m{\dimexpr 0.75in+0\tabcolsep}}{\textcolor[HTML]{000000}{\fontsize{11}{11}\selectfont{0.254}}} \\





\multicolumn{1}{>{\raggedleft}m{\dimexpr 0.75in+0\tabcolsep}}{\textcolor[HTML]{000000}{\fontsize{11}{11}\selectfont{0.4}}} & \multicolumn{1}{>{\raggedleft}m{\dimexpr 0.75in+0\tabcolsep}}{\textcolor[HTML]{000000}{\fontsize{11}{11}\selectfont{0.035}}} & \multicolumn{1}{>{\raggedleft}m{\dimexpr 0.75in+0\tabcolsep}}{\textcolor[HTML]{000000}{\fontsize{11}{11}\selectfont{0.06}}} & \multicolumn{1}{>{\raggedleft}m{\dimexpr 0.75in+0\tabcolsep}}{\textcolor[HTML]{000000}{\fontsize{11}{11}\selectfont{0.002}}} & \multicolumn{1}{>{\raggedleft}m{\dimexpr 0.75in+0\tabcolsep}}{\textcolor[HTML]{000000}{\fontsize{11}{11}\selectfont{0.068}}} \\





\multicolumn{1}{>{\raggedleft}m{\dimexpr 0.75in+0\tabcolsep}}{\textcolor[HTML]{000000}{\fontsize{11}{11}\selectfont{0.5}}} & \multicolumn{1}{>{\raggedleft}m{\dimexpr 0.75in+0\tabcolsep}}{\textcolor[HTML]{000000}{\fontsize{11}{11}\selectfont{0.005}}} & \multicolumn{1}{>{\raggedleft}m{\dimexpr 0.75in+0\tabcolsep}}{\textcolor[HTML]{000000}{\fontsize{11}{11}\selectfont{0.52}}} & \multicolumn{1}{>{\raggedleft}m{\dimexpr 0.75in+0\tabcolsep}}{\textcolor[HTML]{000000}{\fontsize{11}{11}\selectfont{0.002}}} & \multicolumn{1}{>{\raggedleft}m{\dimexpr 0.75in+0\tabcolsep}}{\textcolor[HTML]{000000}{\fontsize{11}{11}\selectfont{0.078}}} \\





\multicolumn{1}{>{\raggedleft}m{\dimexpr 0.75in+0\tabcolsep}}{\textcolor[HTML]{000000}{\fontsize{11}{11}\selectfont{0.6}}} & \multicolumn{1}{>{\raggedleft}m{\dimexpr 0.75in+0\tabcolsep}}{\textcolor[HTML]{000000}{\fontsize{11}{11}\selectfont{0.000}}} & \multicolumn{1}{>{\raggedleft}m{\dimexpr 0.75in+0\tabcolsep}}{\textcolor[HTML]{000000}{\fontsize{11}{11}\selectfont{0.06}}} & \multicolumn{1}{>{\raggedleft}m{\dimexpr 0.75in+0\tabcolsep}}{\textcolor[HTML]{000000}{\fontsize{11}{11}\selectfont{0.000}}} & \multicolumn{1}{>{\raggedleft}m{\dimexpr 0.75in+0\tabcolsep}}{\textcolor[HTML]{000000}{\fontsize{11}{11}\selectfont{0.001}}} \\





\multicolumn{1}{>{\raggedleft}m{\dimexpr 0.75in+0\tabcolsep}}{\textcolor[HTML]{000000}{\fontsize{11}{11}\selectfont{0.7}}} & \multicolumn{1}{>{\raggedleft}m{\dimexpr 0.75in+0\tabcolsep}}{\textcolor[HTML]{000000}{\fontsize{11}{11}\selectfont{0.000}}} & \multicolumn{1}{>{\raggedleft}m{\dimexpr 0.75in+0\tabcolsep}}{\textcolor[HTML]{000000}{\fontsize{11}{11}\selectfont{0.06}}} & \multicolumn{1}{>{\raggedleft}m{\dimexpr 0.75in+0\tabcolsep}}{\textcolor[HTML]{000000}{\fontsize{11}{11}\selectfont{0.000}}} & \multicolumn{1}{>{\raggedleft}m{\dimexpr 0.75in+0\tabcolsep}}{\textcolor[HTML]{000000}{\fontsize{11}{11}\selectfont{0.000}}} \\





\multicolumn{1}{>{\raggedleft}m{\dimexpr 0.75in+0\tabcolsep}}{\textcolor[HTML]{000000}{\fontsize{11}{11}\selectfont{0.8}}} & \multicolumn{1}{>{\raggedleft}m{\dimexpr 0.75in+0\tabcolsep}}{\textcolor[HTML]{000000}{\fontsize{11}{11}\selectfont{0.000}}} & \multicolumn{1}{>{\raggedleft}m{\dimexpr 0.75in+0\tabcolsep}}{\textcolor[HTML]{000000}{\fontsize{11}{11}\selectfont{0.06}}} & \multicolumn{1}{>{\raggedleft}m{\dimexpr 0.75in+0\tabcolsep}}{\textcolor[HTML]{000000}{\fontsize{11}{11}\selectfont{0.000}}} & \multicolumn{1}{>{\raggedleft}m{\dimexpr 0.75in+0\tabcolsep}}{\textcolor[HTML]{000000}{\fontsize{11}{11}\selectfont{0.000}}} \\





\multicolumn{1}{>{\raggedleft}m{\dimexpr 0.75in+0\tabcolsep}}{\textcolor[HTML]{000000}{\fontsize{11}{11}\selectfont{0.9}}} & \multicolumn{1}{>{\raggedleft}m{\dimexpr 0.75in+0\tabcolsep}}{\textcolor[HTML]{000000}{\fontsize{11}{11}\selectfont{0.000}}} & \multicolumn{1}{>{\raggedleft}m{\dimexpr 0.75in+0\tabcolsep}}{\textcolor[HTML]{000000}{\fontsize{11}{11}\selectfont{0.06}}} & \multicolumn{1}{>{\raggedleft}m{\dimexpr 0.75in+0\tabcolsep}}{\textcolor[HTML]{000000}{\fontsize{11}{11}\selectfont{0.000}}} & \multicolumn{1}{>{\raggedleft}m{\dimexpr 0.75in+0\tabcolsep}}{\textcolor[HTML]{000000}{\fontsize{11}{11}\selectfont{0.000}}} \\

\ascline{1.5pt}{666666}{1-5}



\end{longtable*}



\arrayrulecolor[HTML]{000000}

\global\setlength{\arrayrulewidth}{\Oldarrayrulewidth}

\global\setlength{\tabcolsep}{\Oldtabcolsep}

\renewcommand*{\arraystretch}{1}

\begin{Shaded}
\begin{Highlighting}[]
\FunctionTok{ggplot}\NormalTok{(out, }\FunctionTok{aes}\NormalTok{(}\AttributeTok{x=}\NormalTok{p)) }\SpecialCharTok{+} 
  \FunctionTok{geom\_col}\NormalTok{(}\FunctionTok{aes}\NormalTok{(}\AttributeTok{y=}\NormalTok{posterior),}
           \AttributeTok{width=}\NormalTok{.}\DecValTok{05}\NormalTok{,}
           \AttributeTok{color=}\StringTok{"blue"}\NormalTok{,}
           \AttributeTok{fill=}\StringTok{"dodgerblue2"}\NormalTok{,}
           \AttributeTok{alpha=}\NormalTok{.}\DecValTok{6}\NormalTok{) }\SpecialCharTok{+} 
  \FunctionTok{geom\_col}\NormalTok{(}\FunctionTok{aes}\NormalTok{(}\AttributeTok{y=}\NormalTok{prior),}
           \AttributeTok{width=}\NormalTok{.}\DecValTok{05}\NormalTok{,}
           \AttributeTok{color=}\StringTok{"green"}\NormalTok{,}
           \AttributeTok{fill=}\StringTok{"lightgreen"}\NormalTok{,}
           \AttributeTok{alpha=}\NormalTok{.}\DecValTok{6}\NormalTok{) }\SpecialCharTok{+} 
  \FunctionTok{scale\_x\_continuous}\NormalTok{(}\AttributeTok{breaks=}\FunctionTok{seq}\NormalTok{(.}\DecValTok{1}\NormalTok{,.}\DecValTok{9}\NormalTok{,.}\DecValTok{1}\NormalTok{),}
                     \AttributeTok{labels=}\FunctionTok{percent\_format}\NormalTok{(}\AttributeTok{accuracy=}\DecValTok{1}\NormalTok{)) }\SpecialCharTok{+} 
  \FunctionTok{scale\_y\_continuous}\NormalTok{(}\AttributeTok{limits=}\FunctionTok{c}\NormalTok{(}\DecValTok{0}\NormalTok{,.}\DecValTok{55}\NormalTok{),}
                     \AttributeTok{breaks=}\FunctionTok{seq}\NormalTok{(}\DecValTok{0}\NormalTok{,.}\DecValTok{55}\NormalTok{,.}\DecValTok{05}\NormalTok{),}
                     \AttributeTok{labels=}\FunctionTok{percent\_format}\NormalTok{(}\AttributeTok{accuracy=}\DecValTok{1}\NormalTok{)) }\SpecialCharTok{+}
  \FunctionTok{labs}\NormalTok{(}\AttributeTok{x=}\StringTok{"Candidate model value"}\NormalTok{,}
       \AttributeTok{y=}\StringTok{"Probability"}\NormalTok{,}
       \AttributeTok{title=}\StringTok{"Probability of candidate model values\textless{}br\textgreater{}}
\StringTok{       \textless{}span style=\textquotesingle{}color:\#1c86ee;\textquotesingle{}\textgreater{}Posterior distribution\textless{}/span\textgreater{}}
\StringTok{       \textless{}span style=\textquotesingle{}color:\#90ee90;\textquotesingle{}\textgreater{}Prior distribution\textless{}/span\textgreater{}"}\NormalTok{) }\SpecialCharTok{+}
  \FunctionTok{theme}\NormalTok{(}\AttributeTok{axis.title.y=}\FunctionTok{element\_text}\NormalTok{(}\AttributeTok{angle=}\DecValTok{0}\NormalTok{, }\AttributeTok{vjust=}\NormalTok{.}\DecValTok{55}\NormalTok{),}
        \AttributeTok{plot.title=}\FunctionTok{element\_markdown}\NormalTok{())}
\end{Highlighting}
\end{Shaded}

\begin{verbatim}
Warning in grid.Call(C_textBounds, as.graphicsAnnot(x$label), x$x, x$y, : font
width unknown for character 0x20

Warning in grid.Call(C_textBounds, as.graphicsAnnot(x$label), x$x, x$y, : font
width unknown for character 0x20

Warning in grid.Call(C_textBounds, as.graphicsAnnot(x$label), x$x, x$y, : font
width unknown for character 0x20

Warning in grid.Call(C_textBounds, as.graphicsAnnot(x$label), x$x, x$y, : font
width unknown for character 0x20

Warning in grid.Call(C_textBounds, as.graphicsAnnot(x$label), x$x, x$y, : font
width unknown for character 0x20

Warning in grid.Call(C_textBounds, as.graphicsAnnot(x$label), x$x, x$y, : font
width unknown for character 0x20

Warning in grid.Call(C_textBounds, as.graphicsAnnot(x$label), x$x, x$y, : font
width unknown for character 0x20

Warning in grid.Call(C_textBounds, as.graphicsAnnot(x$label), x$x, x$y, : font
width unknown for character 0x20

Warning in grid.Call(C_textBounds, as.graphicsAnnot(x$label), x$x, x$y, : font
width unknown for character 0x20

Warning in grid.Call(C_textBounds, as.graphicsAnnot(x$label), x$x, x$y, : font
width unknown for character 0x20

Warning in grid.Call(C_textBounds, as.graphicsAnnot(x$label), x$x, x$y, : font
width unknown for character 0x20

Warning in grid.Call(C_textBounds, as.graphicsAnnot(x$label), x$x, x$y, : font
width unknown for character 0x20

Warning in grid.Call(C_textBounds, as.graphicsAnnot(x$label), x$x, x$y, : font
width unknown for character 0x20

Warning in grid.Call(C_textBounds, as.graphicsAnnot(x$label), x$x, x$y, : font
width unknown for character 0x20

Warning in grid.Call(C_textBounds, as.graphicsAnnot(x$label), x$x, x$y, : font
width unknown for character 0x20

Warning in grid.Call(C_textBounds, as.graphicsAnnot(x$label), x$x, x$y, : font
width unknown for character 0x20

Warning in grid.Call(C_textBounds, as.graphicsAnnot(x$label), x$x, x$y, : font
width unknown for character 0x20
\end{verbatim}

\begin{verbatim}
Warning in grid.Call.graphics(C_text, as.graphicsAnnot(x$label), x$x, x$y, :
font width unknown for character 0x20

Warning in grid.Call.graphics(C_text, as.graphicsAnnot(x$label), x$x, x$y, :
font width unknown for character 0x20
\end{verbatim}

\begin{figure}[H]

{\centering \includegraphics{rethinking2024wk1_files/figure-pdf/unnamed-chunk-12-1.pdf}

}

\end{figure}

\hypertarget{enumerating-sequences-of-events}{%
\subsection{Enumerating sequences of
events}\label{enumerating-sequences-of-events}}

\hypertarget{the-counting-rule}{%
\subsubsection{The counting rule}\label{the-counting-rule}}

This is just to get our thinking started, because it's nothing other
than counting. If event A could occur three different ways, and event B
could occur two different ways, then the total number of ways events A
and B can occur together is 3x2=6.

This principle generalizes to \emph{m x n} ways for events A and B to
co-occur. This principle holds for sequential events as well. For both
cases, the events must be independent and not change in any way across
the course of the calculations.

\hypertarget{permutations}{%
\subsubsection{Permutations}\label{permutations}}

A permutation is a rearrangement of the order of a sequence. Consider
the letters \emph{a, b, c, d}. How many different ways can they be
ordered?

We can brute force this.

\begin{Shaded}
\begin{Highlighting}[]
\NormalTok{per }\OtherTok{\textless{}{-}} \FunctionTok{read\_excel}\NormalTok{(}\StringTok{"permutation practice.xlsx"}\NormalTok{,}
                  \AttributeTok{sheet=}\StringTok{"permutations"}\NormalTok{)}

\FunctionTok{flextable}\NormalTok{(per)}
\end{Highlighting}
\end{Shaded}

\global\setlength{\Oldarrayrulewidth}{\arrayrulewidth}

\global\setlength{\Oldtabcolsep}{\tabcolsep}

\setlength{\tabcolsep}{0pt}

\renewcommand*{\arraystretch}{1.5}



\providecommand{\ascline}[3]{\noalign{\global\arrayrulewidth #1}\arrayrulecolor[HTML]{#2}\cline{#3}}

\begin{longtable*}[c]{|p{0.75in}|p{0.75in}|p{0.75in}|p{0.75in}}



\ascline{1.5pt}{666666}{1-4}

\multicolumn{1}{>{\raggedright}m{\dimexpr 0.75in+0\tabcolsep}}{\textcolor[HTML]{000000}{\fontsize{11}{11}\selectfont{A}}} & \multicolumn{1}{>{\raggedright}m{\dimexpr 0.75in+0\tabcolsep}}{\textcolor[HTML]{000000}{\fontsize{11}{11}\selectfont{B}}} & \multicolumn{1}{>{\raggedright}m{\dimexpr 0.75in+0\tabcolsep}}{\textcolor[HTML]{000000}{\fontsize{11}{11}\selectfont{C}}} & \multicolumn{1}{>{\raggedright}m{\dimexpr 0.75in+0\tabcolsep}}{\textcolor[HTML]{000000}{\fontsize{11}{11}\selectfont{D}}} \\

\ascline{1.5pt}{666666}{1-4}\endfirsthead 

\ascline{1.5pt}{666666}{1-4}

\multicolumn{1}{>{\raggedright}m{\dimexpr 0.75in+0\tabcolsep}}{\textcolor[HTML]{000000}{\fontsize{11}{11}\selectfont{A}}} & \multicolumn{1}{>{\raggedright}m{\dimexpr 0.75in+0\tabcolsep}}{\textcolor[HTML]{000000}{\fontsize{11}{11}\selectfont{B}}} & \multicolumn{1}{>{\raggedright}m{\dimexpr 0.75in+0\tabcolsep}}{\textcolor[HTML]{000000}{\fontsize{11}{11}\selectfont{C}}} & \multicolumn{1}{>{\raggedright}m{\dimexpr 0.75in+0\tabcolsep}}{\textcolor[HTML]{000000}{\fontsize{11}{11}\selectfont{D}}} \\

\ascline{1.5pt}{666666}{1-4}\endhead



\multicolumn{1}{>{\raggedright}m{\dimexpr 0.75in+0\tabcolsep}}{\textcolor[HTML]{000000}{\fontsize{11}{11}\selectfont{abcd}}} & \multicolumn{1}{>{\raggedright}m{\dimexpr 0.75in+0\tabcolsep}}{\textcolor[HTML]{000000}{\fontsize{11}{11}\selectfont{bacd}}} & \multicolumn{1}{>{\raggedright}m{\dimexpr 0.75in+0\tabcolsep}}{\textcolor[HTML]{000000}{\fontsize{11}{11}\selectfont{cabd}}} & \multicolumn{1}{>{\raggedright}m{\dimexpr 0.75in+0\tabcolsep}}{\textcolor[HTML]{000000}{\fontsize{11}{11}\selectfont{dabc}}} \\





\multicolumn{1}{>{\raggedright}m{\dimexpr 0.75in+0\tabcolsep}}{\textcolor[HTML]{000000}{\fontsize{11}{11}\selectfont{abdc}}} & \multicolumn{1}{>{\raggedright}m{\dimexpr 0.75in+0\tabcolsep}}{\textcolor[HTML]{000000}{\fontsize{11}{11}\selectfont{badc}}} & \multicolumn{1}{>{\raggedright}m{\dimexpr 0.75in+0\tabcolsep}}{\textcolor[HTML]{000000}{\fontsize{11}{11}\selectfont{cadb}}} & \multicolumn{1}{>{\raggedright}m{\dimexpr 0.75in+0\tabcolsep}}{\textcolor[HTML]{000000}{\fontsize{11}{11}\selectfont{dacb}}} \\





\multicolumn{1}{>{\raggedright}m{\dimexpr 0.75in+0\tabcolsep}}{\textcolor[HTML]{000000}{\fontsize{11}{11}\selectfont{acbd}}} & \multicolumn{1}{>{\raggedright}m{\dimexpr 0.75in+0\tabcolsep}}{\textcolor[HTML]{000000}{\fontsize{11}{11}\selectfont{bcad}}} & \multicolumn{1}{>{\raggedright}m{\dimexpr 0.75in+0\tabcolsep}}{\textcolor[HTML]{000000}{\fontsize{11}{11}\selectfont{cbad}}} & \multicolumn{1}{>{\raggedright}m{\dimexpr 0.75in+0\tabcolsep}}{\textcolor[HTML]{000000}{\fontsize{11}{11}\selectfont{dbac}}} \\





\multicolumn{1}{>{\raggedright}m{\dimexpr 0.75in+0\tabcolsep}}{\textcolor[HTML]{000000}{\fontsize{11}{11}\selectfont{acdb}}} & \multicolumn{1}{>{\raggedright}m{\dimexpr 0.75in+0\tabcolsep}}{\textcolor[HTML]{000000}{\fontsize{11}{11}\selectfont{bcda}}} & \multicolumn{1}{>{\raggedright}m{\dimexpr 0.75in+0\tabcolsep}}{\textcolor[HTML]{000000}{\fontsize{11}{11}\selectfont{cbda}}} & \multicolumn{1}{>{\raggedright}m{\dimexpr 0.75in+0\tabcolsep}}{\textcolor[HTML]{000000}{\fontsize{11}{11}\selectfont{dbca}}} \\





\multicolumn{1}{>{\raggedright}m{\dimexpr 0.75in+0\tabcolsep}}{\textcolor[HTML]{000000}{\fontsize{11}{11}\selectfont{adbc}}} & \multicolumn{1}{>{\raggedright}m{\dimexpr 0.75in+0\tabcolsep}}{\textcolor[HTML]{000000}{\fontsize{11}{11}\selectfont{bdac}}} & \multicolumn{1}{>{\raggedright}m{\dimexpr 0.75in+0\tabcolsep}}{\textcolor[HTML]{000000}{\fontsize{11}{11}\selectfont{cdab}}} & \multicolumn{1}{>{\raggedright}m{\dimexpr 0.75in+0\tabcolsep}}{\textcolor[HTML]{000000}{\fontsize{11}{11}\selectfont{dcab}}} \\





\multicolumn{1}{>{\raggedright}m{\dimexpr 0.75in+0\tabcolsep}}{\textcolor[HTML]{000000}{\fontsize{11}{11}\selectfont{adcb}}} & \multicolumn{1}{>{\raggedright}m{\dimexpr 0.75in+0\tabcolsep}}{\textcolor[HTML]{000000}{\fontsize{11}{11}\selectfont{bdca}}} & \multicolumn{1}{>{\raggedright}m{\dimexpr 0.75in+0\tabcolsep}}{\textcolor[HTML]{000000}{\fontsize{11}{11}\selectfont{cdba}}} & \multicolumn{1}{>{\raggedright}m{\dimexpr 0.75in+0\tabcolsep}}{\textcolor[HTML]{000000}{\fontsize{11}{11}\selectfont{dcba}}} \\

\ascline{1.5pt}{666666}{1-4}



\end{longtable*}



\arrayrulecolor[HTML]{000000}

\global\setlength{\arrayrulewidth}{\Oldarrayrulewidth}

\global\setlength{\tabcolsep}{\Oldtabcolsep}

\renewcommand*{\arraystretch}{1}

Each of the four starting letters has six possible orderings.

4x6=24

Another derivation would be to note that there are four possible
selections of the first letter. After the first letter has been
selected, three possible letters remain. After the first two letters are
selected, two letters remain. After three letters have been selected,
there is only one letter left.

By the counting rule, the number of possible orderings is 4x3x2x1 = 24

Note that this is \emph{n(n-1)(n-2)\ldots{} = n!}

4! = 24

\hypertarget{permutation-with-duplicates}{%
\paragraph{Permutation with
duplicates}\label{permutation-with-duplicates}}

What if there is repetition in the orderings? How many different ordered
arrangements are possible with the sequence s, e, l, l?

\begin{Shaded}
\begin{Highlighting}[]
\FunctionTok{here}\NormalTok{()}
\end{Highlighting}
\end{Shaded}

\begin{verbatim}
[1] "C:/Users/dan.killian/Documents/StatisticalRethinking"
\end{verbatim}

\begin{Shaded}
\begin{Highlighting}[]
\FunctionTok{include\_graphics}\NormalTok{(}\StringTok{"perms with dupes.png"}\NormalTok{)}
\end{Highlighting}
\end{Shaded}

\begin{figure}[H]

{\centering \includegraphics[width=1.81in,height=\textheight]{perms with dupes.png}

}

\end{figure}

There are 4! = 24 different permutations, if the items were unique

There are 2! = 2 repeating letter combinations

The unique orderings are \emph{4! / 2! = 12}

Another example: How many different letter arrangements can be formed
from the letters P E P P E R ?

There are \emph{6! = 720} permutations if the items were unique

There are \emph{3! = 6} repeating combinations of the letter P

There are \emph{2! = 2} repeating combinations of the letter E

There are \emph{3!2! = 12} combinations of P and E duplicates

The number of unique orderings is \emph{6! / 3!2! = 6x5x4 / 2x1 = 60}
unique orderings

\hypertarget{permuted-groups}{%
\paragraph{Permuted groups}\label{permuted-groups}}

How many groups of three can be chosen from a set of five?

\begin{Shaded}
\begin{Highlighting}[]
\NormalTok{pg }\OtherTok{\textless{}{-}} \FunctionTok{data.frame}\NormalTok{(}\AttributeTok{num=}\DecValTok{1}\SpecialCharTok{:}\DecValTok{12}\NormalTok{,}
                 \AttributeTok{A =} \FunctionTok{c}\NormalTok{(}\StringTok{"abc"}\NormalTok{,}
                       \StringTok{"abd"}\NormalTok{,}
                       \StringTok{"abe"}\NormalTok{,}
                       \StringTok{"acb"}\NormalTok{,}
                       \StringTok{"acd"}\NormalTok{,}
                       \StringTok{"ace"}\NormalTok{,}
                       \StringTok{"adb"}\NormalTok{,}
                       \StringTok{"adc"}\NormalTok{,}
                       \StringTok{"ade"}\NormalTok{,}
                       \StringTok{"aeb"}\NormalTok{,}
                       \StringTok{"aec"}\NormalTok{,}
                       \StringTok{"aed"}\NormalTok{),}
                 \AttributeTok{B=}\ConstantTok{NA}\NormalTok{, }\AttributeTok{C=}\ConstantTok{NA}\NormalTok{, }\AttributeTok{D=}\ConstantTok{NA}\NormalTok{, }\AttributeTok{E=}\ConstantTok{NA}\NormalTok{)}
\FunctionTok{flextable}\NormalTok{(pg)}
\end{Highlighting}
\end{Shaded}

\global\setlength{\Oldarrayrulewidth}{\arrayrulewidth}

\global\setlength{\Oldtabcolsep}{\tabcolsep}

\setlength{\tabcolsep}{0pt}

\renewcommand*{\arraystretch}{1.5}



\providecommand{\ascline}[3]{\noalign{\global\arrayrulewidth #1}\arrayrulecolor[HTML]{#2}\cline{#3}}

\begin{longtable*}[c]{|p{0.75in}|p{0.75in}|p{0.75in}|p{0.75in}|p{0.75in}|p{0.75in}}



\ascline{1.5pt}{666666}{1-6}

\multicolumn{1}{>{\raggedleft}m{\dimexpr 0.75in+0\tabcolsep}}{\textcolor[HTML]{000000}{\fontsize{11}{11}\selectfont{num}}} & \multicolumn{1}{>{\raggedright}m{\dimexpr 0.75in+0\tabcolsep}}{\textcolor[HTML]{000000}{\fontsize{11}{11}\selectfont{A}}} & \multicolumn{1}{>{\raggedleft}m{\dimexpr 0.75in+0\tabcolsep}}{\textcolor[HTML]{000000}{\fontsize{11}{11}\selectfont{B}}} & \multicolumn{1}{>{\raggedleft}m{\dimexpr 0.75in+0\tabcolsep}}{\textcolor[HTML]{000000}{\fontsize{11}{11}\selectfont{C}}} & \multicolumn{1}{>{\raggedleft}m{\dimexpr 0.75in+0\tabcolsep}}{\textcolor[HTML]{000000}{\fontsize{11}{11}\selectfont{D}}} & \multicolumn{1}{>{\raggedleft}m{\dimexpr 0.75in+0\tabcolsep}}{\textcolor[HTML]{000000}{\fontsize{11}{11}\selectfont{E}}} \\

\ascline{1.5pt}{666666}{1-6}\endfirsthead 

\ascline{1.5pt}{666666}{1-6}

\multicolumn{1}{>{\raggedleft}m{\dimexpr 0.75in+0\tabcolsep}}{\textcolor[HTML]{000000}{\fontsize{11}{11}\selectfont{num}}} & \multicolumn{1}{>{\raggedright}m{\dimexpr 0.75in+0\tabcolsep}}{\textcolor[HTML]{000000}{\fontsize{11}{11}\selectfont{A}}} & \multicolumn{1}{>{\raggedleft}m{\dimexpr 0.75in+0\tabcolsep}}{\textcolor[HTML]{000000}{\fontsize{11}{11}\selectfont{B}}} & \multicolumn{1}{>{\raggedleft}m{\dimexpr 0.75in+0\tabcolsep}}{\textcolor[HTML]{000000}{\fontsize{11}{11}\selectfont{C}}} & \multicolumn{1}{>{\raggedleft}m{\dimexpr 0.75in+0\tabcolsep}}{\textcolor[HTML]{000000}{\fontsize{11}{11}\selectfont{D}}} & \multicolumn{1}{>{\raggedleft}m{\dimexpr 0.75in+0\tabcolsep}}{\textcolor[HTML]{000000}{\fontsize{11}{11}\selectfont{E}}} \\

\ascline{1.5pt}{666666}{1-6}\endhead



\multicolumn{1}{>{\raggedleft}m{\dimexpr 0.75in+0\tabcolsep}}{\textcolor[HTML]{000000}{\fontsize{11}{11}\selectfont{1}}} & \multicolumn{1}{>{\raggedright}m{\dimexpr 0.75in+0\tabcolsep}}{\textcolor[HTML]{000000}{\fontsize{11}{11}\selectfont{abc}}} & \multicolumn{1}{>{\raggedleft}m{\dimexpr 0.75in+0\tabcolsep}}{\textcolor[HTML]{000000}{\fontsize{11}{11}\selectfont{}}} & \multicolumn{1}{>{\raggedleft}m{\dimexpr 0.75in+0\tabcolsep}}{\textcolor[HTML]{000000}{\fontsize{11}{11}\selectfont{}}} & \multicolumn{1}{>{\raggedleft}m{\dimexpr 0.75in+0\tabcolsep}}{\textcolor[HTML]{000000}{\fontsize{11}{11}\selectfont{}}} & \multicolumn{1}{>{\raggedleft}m{\dimexpr 0.75in+0\tabcolsep}}{\textcolor[HTML]{000000}{\fontsize{11}{11}\selectfont{}}} \\





\multicolumn{1}{>{\raggedleft}m{\dimexpr 0.75in+0\tabcolsep}}{\textcolor[HTML]{000000}{\fontsize{11}{11}\selectfont{2}}} & \multicolumn{1}{>{\raggedright}m{\dimexpr 0.75in+0\tabcolsep}}{\textcolor[HTML]{000000}{\fontsize{11}{11}\selectfont{abd}}} & \multicolumn{1}{>{\raggedleft}m{\dimexpr 0.75in+0\tabcolsep}}{\textcolor[HTML]{000000}{\fontsize{11}{11}\selectfont{}}} & \multicolumn{1}{>{\raggedleft}m{\dimexpr 0.75in+0\tabcolsep}}{\textcolor[HTML]{000000}{\fontsize{11}{11}\selectfont{}}} & \multicolumn{1}{>{\raggedleft}m{\dimexpr 0.75in+0\tabcolsep}}{\textcolor[HTML]{000000}{\fontsize{11}{11}\selectfont{}}} & \multicolumn{1}{>{\raggedleft}m{\dimexpr 0.75in+0\tabcolsep}}{\textcolor[HTML]{000000}{\fontsize{11}{11}\selectfont{}}} \\





\multicolumn{1}{>{\raggedleft}m{\dimexpr 0.75in+0\tabcolsep}}{\textcolor[HTML]{000000}{\fontsize{11}{11}\selectfont{3}}} & \multicolumn{1}{>{\raggedright}m{\dimexpr 0.75in+0\tabcolsep}}{\textcolor[HTML]{000000}{\fontsize{11}{11}\selectfont{abe}}} & \multicolumn{1}{>{\raggedleft}m{\dimexpr 0.75in+0\tabcolsep}}{\textcolor[HTML]{000000}{\fontsize{11}{11}\selectfont{}}} & \multicolumn{1}{>{\raggedleft}m{\dimexpr 0.75in+0\tabcolsep}}{\textcolor[HTML]{000000}{\fontsize{11}{11}\selectfont{}}} & \multicolumn{1}{>{\raggedleft}m{\dimexpr 0.75in+0\tabcolsep}}{\textcolor[HTML]{000000}{\fontsize{11}{11}\selectfont{}}} & \multicolumn{1}{>{\raggedleft}m{\dimexpr 0.75in+0\tabcolsep}}{\textcolor[HTML]{000000}{\fontsize{11}{11}\selectfont{}}} \\





\multicolumn{1}{>{\raggedleft}m{\dimexpr 0.75in+0\tabcolsep}}{\textcolor[HTML]{000000}{\fontsize{11}{11}\selectfont{4}}} & \multicolumn{1}{>{\raggedright}m{\dimexpr 0.75in+0\tabcolsep}}{\textcolor[HTML]{000000}{\fontsize{11}{11}\selectfont{acb}}} & \multicolumn{1}{>{\raggedleft}m{\dimexpr 0.75in+0\tabcolsep}}{\textcolor[HTML]{000000}{\fontsize{11}{11}\selectfont{}}} & \multicolumn{1}{>{\raggedleft}m{\dimexpr 0.75in+0\tabcolsep}}{\textcolor[HTML]{000000}{\fontsize{11}{11}\selectfont{}}} & \multicolumn{1}{>{\raggedleft}m{\dimexpr 0.75in+0\tabcolsep}}{\textcolor[HTML]{000000}{\fontsize{11}{11}\selectfont{}}} & \multicolumn{1}{>{\raggedleft}m{\dimexpr 0.75in+0\tabcolsep}}{\textcolor[HTML]{000000}{\fontsize{11}{11}\selectfont{}}} \\





\multicolumn{1}{>{\raggedleft}m{\dimexpr 0.75in+0\tabcolsep}}{\textcolor[HTML]{000000}{\fontsize{11}{11}\selectfont{5}}} & \multicolumn{1}{>{\raggedright}m{\dimexpr 0.75in+0\tabcolsep}}{\textcolor[HTML]{000000}{\fontsize{11}{11}\selectfont{acd}}} & \multicolumn{1}{>{\raggedleft}m{\dimexpr 0.75in+0\tabcolsep}}{\textcolor[HTML]{000000}{\fontsize{11}{11}\selectfont{}}} & \multicolumn{1}{>{\raggedleft}m{\dimexpr 0.75in+0\tabcolsep}}{\textcolor[HTML]{000000}{\fontsize{11}{11}\selectfont{}}} & \multicolumn{1}{>{\raggedleft}m{\dimexpr 0.75in+0\tabcolsep}}{\textcolor[HTML]{000000}{\fontsize{11}{11}\selectfont{}}} & \multicolumn{1}{>{\raggedleft}m{\dimexpr 0.75in+0\tabcolsep}}{\textcolor[HTML]{000000}{\fontsize{11}{11}\selectfont{}}} \\





\multicolumn{1}{>{\raggedleft}m{\dimexpr 0.75in+0\tabcolsep}}{\textcolor[HTML]{000000}{\fontsize{11}{11}\selectfont{6}}} & \multicolumn{1}{>{\raggedright}m{\dimexpr 0.75in+0\tabcolsep}}{\textcolor[HTML]{000000}{\fontsize{11}{11}\selectfont{ace}}} & \multicolumn{1}{>{\raggedleft}m{\dimexpr 0.75in+0\tabcolsep}}{\textcolor[HTML]{000000}{\fontsize{11}{11}\selectfont{}}} & \multicolumn{1}{>{\raggedleft}m{\dimexpr 0.75in+0\tabcolsep}}{\textcolor[HTML]{000000}{\fontsize{11}{11}\selectfont{}}} & \multicolumn{1}{>{\raggedleft}m{\dimexpr 0.75in+0\tabcolsep}}{\textcolor[HTML]{000000}{\fontsize{11}{11}\selectfont{}}} & \multicolumn{1}{>{\raggedleft}m{\dimexpr 0.75in+0\tabcolsep}}{\textcolor[HTML]{000000}{\fontsize{11}{11}\selectfont{}}} \\





\multicolumn{1}{>{\raggedleft}m{\dimexpr 0.75in+0\tabcolsep}}{\textcolor[HTML]{000000}{\fontsize{11}{11}\selectfont{7}}} & \multicolumn{1}{>{\raggedright}m{\dimexpr 0.75in+0\tabcolsep}}{\textcolor[HTML]{000000}{\fontsize{11}{11}\selectfont{adb}}} & \multicolumn{1}{>{\raggedleft}m{\dimexpr 0.75in+0\tabcolsep}}{\textcolor[HTML]{000000}{\fontsize{11}{11}\selectfont{}}} & \multicolumn{1}{>{\raggedleft}m{\dimexpr 0.75in+0\tabcolsep}}{\textcolor[HTML]{000000}{\fontsize{11}{11}\selectfont{}}} & \multicolumn{1}{>{\raggedleft}m{\dimexpr 0.75in+0\tabcolsep}}{\textcolor[HTML]{000000}{\fontsize{11}{11}\selectfont{}}} & \multicolumn{1}{>{\raggedleft}m{\dimexpr 0.75in+0\tabcolsep}}{\textcolor[HTML]{000000}{\fontsize{11}{11}\selectfont{}}} \\





\multicolumn{1}{>{\raggedleft}m{\dimexpr 0.75in+0\tabcolsep}}{\textcolor[HTML]{000000}{\fontsize{11}{11}\selectfont{8}}} & \multicolumn{1}{>{\raggedright}m{\dimexpr 0.75in+0\tabcolsep}}{\textcolor[HTML]{000000}{\fontsize{11}{11}\selectfont{adc}}} & \multicolumn{1}{>{\raggedleft}m{\dimexpr 0.75in+0\tabcolsep}}{\textcolor[HTML]{000000}{\fontsize{11}{11}\selectfont{}}} & \multicolumn{1}{>{\raggedleft}m{\dimexpr 0.75in+0\tabcolsep}}{\textcolor[HTML]{000000}{\fontsize{11}{11}\selectfont{}}} & \multicolumn{1}{>{\raggedleft}m{\dimexpr 0.75in+0\tabcolsep}}{\textcolor[HTML]{000000}{\fontsize{11}{11}\selectfont{}}} & \multicolumn{1}{>{\raggedleft}m{\dimexpr 0.75in+0\tabcolsep}}{\textcolor[HTML]{000000}{\fontsize{11}{11}\selectfont{}}} \\





\multicolumn{1}{>{\raggedleft}m{\dimexpr 0.75in+0\tabcolsep}}{\textcolor[HTML]{000000}{\fontsize{11}{11}\selectfont{9}}} & \multicolumn{1}{>{\raggedright}m{\dimexpr 0.75in+0\tabcolsep}}{\textcolor[HTML]{000000}{\fontsize{11}{11}\selectfont{ade}}} & \multicolumn{1}{>{\raggedleft}m{\dimexpr 0.75in+0\tabcolsep}}{\textcolor[HTML]{000000}{\fontsize{11}{11}\selectfont{}}} & \multicolumn{1}{>{\raggedleft}m{\dimexpr 0.75in+0\tabcolsep}}{\textcolor[HTML]{000000}{\fontsize{11}{11}\selectfont{}}} & \multicolumn{1}{>{\raggedleft}m{\dimexpr 0.75in+0\tabcolsep}}{\textcolor[HTML]{000000}{\fontsize{11}{11}\selectfont{}}} & \multicolumn{1}{>{\raggedleft}m{\dimexpr 0.75in+0\tabcolsep}}{\textcolor[HTML]{000000}{\fontsize{11}{11}\selectfont{}}} \\





\multicolumn{1}{>{\raggedleft}m{\dimexpr 0.75in+0\tabcolsep}}{\textcolor[HTML]{000000}{\fontsize{11}{11}\selectfont{10}}} & \multicolumn{1}{>{\raggedright}m{\dimexpr 0.75in+0\tabcolsep}}{\textcolor[HTML]{000000}{\fontsize{11}{11}\selectfont{aeb}}} & \multicolumn{1}{>{\raggedleft}m{\dimexpr 0.75in+0\tabcolsep}}{\textcolor[HTML]{000000}{\fontsize{11}{11}\selectfont{}}} & \multicolumn{1}{>{\raggedleft}m{\dimexpr 0.75in+0\tabcolsep}}{\textcolor[HTML]{000000}{\fontsize{11}{11}\selectfont{}}} & \multicolumn{1}{>{\raggedleft}m{\dimexpr 0.75in+0\tabcolsep}}{\textcolor[HTML]{000000}{\fontsize{11}{11}\selectfont{}}} & \multicolumn{1}{>{\raggedleft}m{\dimexpr 0.75in+0\tabcolsep}}{\textcolor[HTML]{000000}{\fontsize{11}{11}\selectfont{}}} \\





\multicolumn{1}{>{\raggedleft}m{\dimexpr 0.75in+0\tabcolsep}}{\textcolor[HTML]{000000}{\fontsize{11}{11}\selectfont{11}}} & \multicolumn{1}{>{\raggedright}m{\dimexpr 0.75in+0\tabcolsep}}{\textcolor[HTML]{000000}{\fontsize{11}{11}\selectfont{aec}}} & \multicolumn{1}{>{\raggedleft}m{\dimexpr 0.75in+0\tabcolsep}}{\textcolor[HTML]{000000}{\fontsize{11}{11}\selectfont{}}} & \multicolumn{1}{>{\raggedleft}m{\dimexpr 0.75in+0\tabcolsep}}{\textcolor[HTML]{000000}{\fontsize{11}{11}\selectfont{}}} & \multicolumn{1}{>{\raggedleft}m{\dimexpr 0.75in+0\tabcolsep}}{\textcolor[HTML]{000000}{\fontsize{11}{11}\selectfont{}}} & \multicolumn{1}{>{\raggedleft}m{\dimexpr 0.75in+0\tabcolsep}}{\textcolor[HTML]{000000}{\fontsize{11}{11}\selectfont{}}} \\





\multicolumn{1}{>{\raggedleft}m{\dimexpr 0.75in+0\tabcolsep}}{\textcolor[HTML]{000000}{\fontsize{11}{11}\selectfont{12}}} & \multicolumn{1}{>{\raggedright}m{\dimexpr 0.75in+0\tabcolsep}}{\textcolor[HTML]{000000}{\fontsize{11}{11}\selectfont{aed}}} & \multicolumn{1}{>{\raggedleft}m{\dimexpr 0.75in+0\tabcolsep}}{\textcolor[HTML]{000000}{\fontsize{11}{11}\selectfont{}}} & \multicolumn{1}{>{\raggedleft}m{\dimexpr 0.75in+0\tabcolsep}}{\textcolor[HTML]{000000}{\fontsize{11}{11}\selectfont{}}} & \multicolumn{1}{>{\raggedleft}m{\dimexpr 0.75in+0\tabcolsep}}{\textcolor[HTML]{000000}{\fontsize{11}{11}\selectfont{}}} & \multicolumn{1}{>{\raggedleft}m{\dimexpr 0.75in+0\tabcolsep}}{\textcolor[HTML]{000000}{\fontsize{11}{11}\selectfont{}}} \\

\ascline{1.5pt}{666666}{1-6}



\end{longtable*}



\arrayrulecolor[HTML]{000000}

\global\setlength{\arrayrulewidth}{\Oldarrayrulewidth}

\global\setlength{\tabcolsep}{\Oldtabcolsep}

\renewcommand*{\arraystretch}{1}

There are \emph{12x5=60} possible groups of three

Another derivation would be to say that there are five possibilities for
the first letter, four possibilities for the second letter, and three
possibilities for the third letter.

\emph{5x4x3=60} possible groups of three

This can be generalized as \(P(N, K) = \frac{N!}{ (N-K)!}\)

Using the example, \(P(N,K)=\frac{5!}{(5-3!)}=60\) possible groups of
three

\hypertarget{combinations}{%
\subsubsection{Combinations}\label{combinations}}

Continuing the example above, how many groups of three items from a set
of five can we select, where the ordering does not matter? We would
express this as choose three from five, or choose \emph{K} from
\emph{N}, which is written as \(\binom{N}{K}\).

If ordering does not matter, then the groups \emph{a,b,c} and
\emph{a,c,b} would be considered duplicates. Each group can be ordered
in \emph{K!} ways and these would be duplicates. That leaves us with:

\(\binom{N}{K}=\frac{P(N,K)}{K!}\)

Which is more commonly written after substituting in the expression for
\emph{P(N,K)}:

\(\binom{N}{K}=\frac{N!}{(N-K)!K!}\)

You could also do a brute force example:

How many groups of three can be formed from the letters a, b, c, d?

\begin{Shaded}
\begin{Highlighting}[]
\FunctionTok{include\_graphics}\NormalTok{(}\StringTok{"comb2.png"}\NormalTok{)}
\end{Highlighting}
\end{Shaded}

\begin{figure}[H]

{\centering \includegraphics[width=1.77in,height=\textheight]{comb2.png}

}

\end{figure}

There are \emph{4!=24} possible permutations

If ordering does not matter, then there are
\(P(N,K)=\frac{N!}{ (N-K)!}=\frac{4!}{(2-2)!}=12\) possible groups of
two to be selected from a set of four.

If ordering does matter, then there are
\(\binom{N}{K}=\frac{P(N,K)}{K!}=\frac{N!}{(N-K)!K!}=\frac{4!}{(4-2)!2!}=6\)

\hypertarget{randomization-inference}{%
\subsubsection{Randomization inference}\label{randomization-inference}}

Permutations may be used to generate test statistics.

\hypertarget{homework}{%
\subsection{Homework}\label{homework}}

\begin{enumerate}
\def\labelenumi{\arabic{enumi}.}
\item
  Suppose the globe tossing data had turned out to be 3 water and 11
  land
\item
\item
  \href{https://www.coursera.org/learn/bayesian/lecture/cOdqz/inference-for-a-proportion-bayesian-approach}{here}.
  Construct the posterior distribution.

  Slide 50 starts coding the estimator using the example of a single
  sample of 6 waters and 3 lands.

  Slide 53 provides a function to simulate a set of globe tosses of any
  probability and number of tosses.

  Slide 59 provides a function to compute the posterior, that includes
  the simulation function as an argument.

  Slide 90 simulates the posterior predictive distribution

  In reviewing this code, note that the simulation will give you a
  probabilistic response, while the homework establishes an exact sample
  of 3 water and 11 land. Let's modify the code on slide 59 to accept
  this exact sample.

\begin{Shaded}
\begin{Highlighting}[]
\NormalTok{compute\_posterior }\OtherTok{\textless{}{-}} \ControlFlowTok{function}\NormalTok{(W, L,}
                              \AttributeTok{poss=}\FunctionTok{seq}\NormalTok{(}\DecValTok{0}\NormalTok{,}\DecValTok{1}\NormalTok{,}\AttributeTok{len=}\DecValTok{11}\NormalTok{)) \{}
\NormalTok{  ways }\OtherTok{\textless{}{-}} \FunctionTok{sapply}\NormalTok{(poss, }\ControlFlowTok{function}\NormalTok{(q) (q}\SpecialCharTok{*}\DecValTok{4}\NormalTok{)}\SpecialCharTok{\^{}}\NormalTok{W }\SpecialCharTok{*}\NormalTok{ (}\DecValTok{4{-}4}\SpecialCharTok{*}\NormalTok{q)}\SpecialCharTok{\^{}}\NormalTok{L)}
\NormalTok{  post }\OtherTok{\textless{}{-}}\NormalTok{ ways}\SpecialCharTok{/}\FunctionTok{sum}\NormalTok{(ways)}
\NormalTok{  bars }\OtherTok{\textless{}{-}} \FunctionTok{sapply}\NormalTok{(post, }\ControlFlowTok{function}\NormalTok{(q) }\FunctionTok{make\_bar}\NormalTok{(q))}
  \FunctionTok{data.frame}\NormalTok{(poss, ways, post, bars)}
\NormalTok{\}  }

\NormalTok{out }\OtherTok{\textless{}{-}} \FunctionTok{compute\_posterior}\NormalTok{(}\AttributeTok{W=}\DecValTok{3}\NormalTok{,}
                  \AttributeTok{L=}\DecValTok{11}\NormalTok{)}
\FunctionTok{flextable}\NormalTok{(out)}
\end{Highlighting}
\end{Shaded}

  \global\setlength{\Oldarrayrulewidth}{\arrayrulewidth}

  \global\setlength{\Oldtabcolsep}{\tabcolsep}

  \setlength{\tabcolsep}{0pt}

  \renewcommand*{\arraystretch}{1.5}



  \providecommand{\ascline}[3]{\noalign{\global\arrayrulewidth #1}\arrayrulecolor[HTML]{#2}\cline{#3}}

  \begin{longtable*}[c]{|p{0.75in}|p{0.75in}|p{0.75in}|p{0.75in}}



  \ascline{1.5pt}{666666}{1-4}

  \multicolumn{1}{>{\raggedleft}m{\dimexpr 0.75in+0\tabcolsep}}{\textcolor[HTML]{000000}{\fontsize{11}{11}\selectfont{poss}}} & \multicolumn{1}{>{\raggedleft}m{\dimexpr 0.75in+0\tabcolsep}}{\textcolor[HTML]{000000}{\fontsize{11}{11}\selectfont{ways}}} & \multicolumn{1}{>{\raggedleft}m{\dimexpr 0.75in+0\tabcolsep}}{\textcolor[HTML]{000000}{\fontsize{11}{11}\selectfont{post}}} & \multicolumn{1}{>{\raggedright}m{\dimexpr 0.75in+0\tabcolsep}}{\textcolor[HTML]{000000}{\fontsize{11}{11}\selectfont{bars}}} \\

  \ascline{1.5pt}{666666}{1-4}\endfirsthead 

  \ascline{1.5pt}{666666}{1-4}

  \multicolumn{1}{>{\raggedleft}m{\dimexpr 0.75in+0\tabcolsep}}{\textcolor[HTML]{000000}{\fontsize{11}{11}\selectfont{poss}}} & \multicolumn{1}{>{\raggedleft}m{\dimexpr 0.75in+0\tabcolsep}}{\textcolor[HTML]{000000}{\fontsize{11}{11}\selectfont{ways}}} & \multicolumn{1}{>{\raggedleft}m{\dimexpr 0.75in+0\tabcolsep}}{\textcolor[HTML]{000000}{\fontsize{11}{11}\selectfont{post}}} & \multicolumn{1}{>{\raggedright}m{\dimexpr 0.75in+0\tabcolsep}}{\textcolor[HTML]{000000}{\fontsize{11}{11}\selectfont{bars}}} \\

  \ascline{1.5pt}{666666}{1-4}\endhead



  \multicolumn{1}{>{\raggedleft}m{\dimexpr 0.75in+0\tabcolsep}}{\textcolor[HTML]{000000}{\fontsize{11}{11}\selectfont{0.0}}} & \multicolumn{1}{>{\raggedleft}m{\dimexpr 0.75in+0\tabcolsep}}{\textcolor[HTML]{000000}{\fontsize{11}{11}\selectfont{0.00000}}} & \multicolumn{1}{>{\raggedleft}m{\dimexpr 0.75in+0\tabcolsep}}{\textcolor[HTML]{000000}{\fontsize{11}{11}\selectfont{0.00000000000}}} & \multicolumn{1}{>{\raggedright}m{\dimexpr 0.75in+0\tabcolsep}}{\textcolor[HTML]{000000}{\fontsize{11}{11}\selectfont{\ \ \ \ \ \ \ \ \ \ \ \ \ \ \ \ \ \ \ \ }}} \\





  \multicolumn{1}{>{\raggedleft}m{\dimexpr 0.75in+0\tabcolsep}}{\textcolor[HTML]{000000}{\fontsize{11}{11}\selectfont{0.1}}} & \multicolumn{1}{>{\raggedleft}m{\dimexpr 0.75in+0\tabcolsep}}{\textcolor[HTML]{000000}{\fontsize{11}{11}\selectfont{84,237.89046}}} & \multicolumn{1}{>{\raggedleft}m{\dimexpr 0.75in+0\tabcolsep}}{\textcolor[HTML]{000000}{\fontsize{11}{11}\selectfont{0.17075463813}}} & \multicolumn{1}{>{\raggedright}m{\dimexpr 0.75in+0\tabcolsep}}{\textcolor[HTML]{000000}{\fontsize{11}{11}\selectfont{\#\#\#\ \ \ \ \ \ \ \ \ \ \ \ \ \ \ \ \ }}} \\





  \multicolumn{1}{>{\raggedleft}m{\dimexpr 0.75in+0\tabcolsep}}{\textcolor[HTML]{000000}{\fontsize{11}{11}\selectfont{0.2}}} & \multicolumn{1}{>{\raggedleft}m{\dimexpr 0.75in+0\tabcolsep}}{\textcolor[HTML]{000000}{\fontsize{11}{11}\selectfont{184,467.44074}}} & \multicolumn{1}{>{\raggedleft}m{\dimexpr 0.75in+0\tabcolsep}}{\textcolor[HTML]{000000}{\fontsize{11}{11}\selectfont{0.37392521248}}} & \multicolumn{1}{>{\raggedright}m{\dimexpr 0.75in+0\tabcolsep}}{\textcolor[HTML]{000000}{\fontsize{11}{11}\selectfont{\#\#\#\#\#\#\#\ \ \ \ \ \ \ \ \ \ \ \ \ }}} \\





  \multicolumn{1}{>{\raggedleft}m{\dimexpr 0.75in+0\tabcolsep}}{\textcolor[HTML]{000000}{\fontsize{11}{11}\selectfont{0.3}}} & \multicolumn{1}{>{\raggedleft}m{\dimexpr 0.75in+0\tabcolsep}}{\textcolor[HTML]{000000}{\fontsize{11}{11}\selectfont{143,311.84360}}} & \multicolumn{1}{>{\raggedleft}m{\dimexpr 0.75in+0\tabcolsep}}{\textcolor[HTML]{000000}{\fontsize{11}{11}\selectfont{0.29050065071}}} & \multicolumn{1}{>{\raggedright}m{\dimexpr 0.75in+0\tabcolsep}}{\textcolor[HTML]{000000}{\fontsize{11}{11}\selectfont{\#\#\#\#\#\#\ \ \ \ \ \ \ \ \ \ \ \ \ \ }}} \\





  \multicolumn{1}{>{\raggedleft}m{\dimexpr 0.75in+0\tabcolsep}}{\textcolor[HTML]{000000}{\fontsize{11}{11}\selectfont{0.4}}} & \multicolumn{1}{>{\raggedleft}m{\dimexpr 0.75in+0\tabcolsep}}{\textcolor[HTML]{000000}{\fontsize{11}{11}\selectfont{62,328.05962}}} & \multicolumn{1}{>{\raggedleft}m{\dimexpr 0.75in+0\tabcolsep}}{\textcolor[HTML]{000000}{\fontsize{11}{11}\selectfont{0.12634225772}}} & \multicolumn{1}{>{\raggedright}m{\dimexpr 0.75in+0\tabcolsep}}{\textcolor[HTML]{000000}{\fontsize{11}{11}\selectfont{\#\#\#\ \ \ \ \ \ \ \ \ \ \ \ \ \ \ \ \ }}} \\





  \multicolumn{1}{>{\raggedleft}m{\dimexpr 0.75in+0\tabcolsep}}{\textcolor[HTML]{000000}{\fontsize{11}{11}\selectfont{0.5}}} & \multicolumn{1}{>{\raggedleft}m{\dimexpr 0.75in+0\tabcolsep}}{\textcolor[HTML]{000000}{\fontsize{11}{11}\selectfont{16,384.00000}}} & \multicolumn{1}{>{\raggedleft}m{\dimexpr 0.75in+0\tabcolsep}}{\textcolor[HTML]{000000}{\fontsize{11}{11}\selectfont{0.03321123043}}} & \multicolumn{1}{>{\raggedright}m{\dimexpr 0.75in+0\tabcolsep}}{\textcolor[HTML]{000000}{\fontsize{11}{11}\selectfont{\#\ \ \ \ \ \ \ \ \ \ \ \ \ \ \ \ \ \ \ }}} \\





  \multicolumn{1}{>{\raggedleft}m{\dimexpr 0.75in+0\tabcolsep}}{\textcolor[HTML]{000000}{\fontsize{11}{11}\selectfont{0.6}}} & \multicolumn{1}{>{\raggedleft}m{\dimexpr 0.75in+0\tabcolsep}}{\textcolor[HTML]{000000}{\fontsize{11}{11}\selectfont{2,431.94380}}} & \multicolumn{1}{>{\raggedleft}m{\dimexpr 0.75in+0\tabcolsep}}{\textcolor[HTML]{000000}{\fontsize{11}{11}\selectfont{0.00492967809}}} & \multicolumn{1}{>{\raggedright}m{\dimexpr 0.75in+0\tabcolsep}}{\textcolor[HTML]{000000}{\fontsize{11}{11}\selectfont{\ \ \ \ \ \ \ \ \ \ \ \ \ \ \ \ \ \ \ \ }}} \\





  \multicolumn{1}{>{\raggedleft}m{\dimexpr 0.75in+0\tabcolsep}}{\textcolor[HTML]{000000}{\fontsize{11}{11}\selectfont{0.7}}} & \multicolumn{1}{>{\raggedleft}m{\dimexpr 0.75in+0\tabcolsep}}{\textcolor[HTML]{000000}{\fontsize{11}{11}\selectfont{163.10520}}} & \multicolumn{1}{>{\raggedleft}m{\dimexpr 0.75in+0\tabcolsep}}{\textcolor[HTML]{000000}{\fontsize{11}{11}\selectfont{0.00033062282}}} & \multicolumn{1}{>{\raggedright}m{\dimexpr 0.75in+0\tabcolsep}}{\textcolor[HTML]{000000}{\fontsize{11}{11}\selectfont{\ \ \ \ \ \ \ \ \ \ \ \ \ \ \ \ \ \ \ \ }}} \\





  \multicolumn{1}{>{\raggedleft}m{\dimexpr 0.75in+0\tabcolsep}}{\textcolor[HTML]{000000}{\fontsize{11}{11}\selectfont{0.8}}} & \multicolumn{1}{>{\raggedleft}m{\dimexpr 0.75in+0\tabcolsep}}{\textcolor[HTML]{000000}{\fontsize{11}{11}\selectfont{2.81475}}} & \multicolumn{1}{>{\raggedleft}m{\dimexpr 0.75in+0\tabcolsep}}{\textcolor[HTML]{000000}{\fontsize{11}{11}\selectfont{0.00000570565}}} & \multicolumn{1}{>{\raggedright}m{\dimexpr 0.75in+0\tabcolsep}}{\textcolor[HTML]{000000}{\fontsize{11}{11}\selectfont{\ \ \ \ \ \ \ \ \ \ \ \ \ \ \ \ \ \ \ \ }}} \\





  \multicolumn{1}{>{\raggedleft}m{\dimexpr 0.75in+0\tabcolsep}}{\textcolor[HTML]{000000}{\fontsize{11}{11}\selectfont{0.9}}} & \multicolumn{1}{>{\raggedleft}m{\dimexpr 0.75in+0\tabcolsep}}{\textcolor[HTML]{000000}{\fontsize{11}{11}\selectfont{0.00196}}} & \multicolumn{1}{>{\raggedleft}m{\dimexpr 0.75in+0\tabcolsep}}{\textcolor[HTML]{000000}{\fontsize{11}{11}\selectfont{0.00000000397}}} & \multicolumn{1}{>{\raggedright}m{\dimexpr 0.75in+0\tabcolsep}}{\textcolor[HTML]{000000}{\fontsize{11}{11}\selectfont{\ \ \ \ \ \ \ \ \ \ \ \ \ \ \ \ \ \ \ \ }}} \\





  \multicolumn{1}{>{\raggedleft}m{\dimexpr 0.75in+0\tabcolsep}}{\textcolor[HTML]{000000}{\fontsize{11}{11}\selectfont{1.0}}} & \multicolumn{1}{>{\raggedleft}m{\dimexpr 0.75in+0\tabcolsep}}{\textcolor[HTML]{000000}{\fontsize{11}{11}\selectfont{0.00000}}} & \multicolumn{1}{>{\raggedleft}m{\dimexpr 0.75in+0\tabcolsep}}{\textcolor[HTML]{000000}{\fontsize{11}{11}\selectfont{0.00000000000}}} & \multicolumn{1}{>{\raggedright}m{\dimexpr 0.75in+0\tabcolsep}}{\textcolor[HTML]{000000}{\fontsize{11}{11}\selectfont{\ \ \ \ \ \ \ \ \ \ \ \ \ \ \ \ \ \ \ \ }}} \\

  \ascline{1.5pt}{666666}{1-4}



  \end{longtable*}



  \arrayrulecolor[HTML]{000000}

  \global\setlength{\arrayrulewidth}{\Oldarrayrulewidth}

  \global\setlength{\tabcolsep}{\Oldtabcolsep}

  \renewcommand*{\arraystretch}{1}

  Just like the example above, plot the posterior distribution

\begin{Shaded}
\begin{Highlighting}[]
\FunctionTok{ggplot}\NormalTok{(out, }\FunctionTok{aes}\NormalTok{(}\AttributeTok{x=}\NormalTok{poss)) }\SpecialCharTok{+} 
  \FunctionTok{geom\_col}\NormalTok{(}\FunctionTok{aes}\NormalTok{(}\AttributeTok{y=}\NormalTok{post),}
           \AttributeTok{width=}\NormalTok{.}\DecValTok{05}\NormalTok{,}
           \AttributeTok{color=}\StringTok{"blue"}\NormalTok{,}
           \AttributeTok{fill=}\StringTok{"dodgerblue2"}\NormalTok{,}
           \AttributeTok{alpha=}\NormalTok{.}\DecValTok{6}\NormalTok{) }\SpecialCharTok{+} 
  \CommentTok{\# geom\_col(aes(y=prior),}
  \CommentTok{\#          width=.05,}
  \CommentTok{\#          color="green",}
  \CommentTok{\#          fill="lightgreen",}
  \CommentTok{\#          alpha=.6) + }
  \FunctionTok{scale\_x\_continuous}\NormalTok{(}\AttributeTok{breaks=}\FunctionTok{seq}\NormalTok{(}\DecValTok{0}\NormalTok{,}\DecValTok{1}\NormalTok{,.}\DecValTok{1}\NormalTok{),}
                     \AttributeTok{labels=}\FunctionTok{percent\_format}\NormalTok{(}\AttributeTok{accuracy=}\DecValTok{1}\NormalTok{)) }\SpecialCharTok{+} 
  \FunctionTok{scale\_y\_continuous}\NormalTok{(}\AttributeTok{limits=}\FunctionTok{c}\NormalTok{(}\DecValTok{0}\NormalTok{,.}\DecValTok{4}\NormalTok{),}
                     \AttributeTok{breaks=}\FunctionTok{seq}\NormalTok{(}\DecValTok{0}\NormalTok{,.}\DecValTok{4}\NormalTok{,.}\DecValTok{05}\NormalTok{),}
                     \AttributeTok{labels=}\FunctionTok{percent\_format}\NormalTok{(}\AttributeTok{accuracy=}\DecValTok{1}\NormalTok{)) }\SpecialCharTok{+}
  \FunctionTok{labs}\NormalTok{(}\AttributeTok{x=}\StringTok{"Candidate model value"}\NormalTok{,}
       \AttributeTok{y=}\StringTok{"Probability"}\NormalTok{,}
       \AttributeTok{title=}\StringTok{"Posterior distribution"}\NormalTok{) }\SpecialCharTok{+}
  \FunctionTok{theme}\NormalTok{(}\AttributeTok{axis.title.y=}\FunctionTok{element\_text}\NormalTok{(}\AttributeTok{angle=}\DecValTok{0}\NormalTok{, }\AttributeTok{vjust=}\NormalTok{.}\DecValTok{55}\NormalTok{))}
\end{Highlighting}
\end{Shaded}

\begin{verbatim}
Warning in grid.Call(C_textBounds, as.graphicsAnnot(x$label), x$x, x$y, : font
width unknown for character 0x20

Warning in grid.Call(C_textBounds, as.graphicsAnnot(x$label), x$x, x$y, : font
width unknown for character 0x20

Warning in grid.Call(C_textBounds, as.graphicsAnnot(x$label), x$x, x$y, : font
width unknown for character 0x20

Warning in grid.Call(C_textBounds, as.graphicsAnnot(x$label), x$x, x$y, : font
width unknown for character 0x20

Warning in grid.Call(C_textBounds, as.graphicsAnnot(x$label), x$x, x$y, : font
width unknown for character 0x20

Warning in grid.Call(C_textBounds, as.graphicsAnnot(x$label), x$x, x$y, : font
width unknown for character 0x20

Warning in grid.Call(C_textBounds, as.graphicsAnnot(x$label), x$x, x$y, : font
width unknown for character 0x20

Warning in grid.Call(C_textBounds, as.graphicsAnnot(x$label), x$x, x$y, : font
width unknown for character 0x20

Warning in grid.Call(C_textBounds, as.graphicsAnnot(x$label), x$x, x$y, : font
width unknown for character 0x20

Warning in grid.Call(C_textBounds, as.graphicsAnnot(x$label), x$x, x$y, : font
width unknown for character 0x20

Warning in grid.Call(C_textBounds, as.graphicsAnnot(x$label), x$x, x$y, : font
width unknown for character 0x20

Warning in grid.Call(C_textBounds, as.graphicsAnnot(x$label), x$x, x$y, : font
width unknown for character 0x20

Warning in grid.Call(C_textBounds, as.graphicsAnnot(x$label), x$x, x$y, : font
width unknown for character 0x20

Warning in grid.Call(C_textBounds, as.graphicsAnnot(x$label), x$x, x$y, : font
width unknown for character 0x20

Warning in grid.Call(C_textBounds, as.graphicsAnnot(x$label), x$x, x$y, : font
width unknown for character 0x20

Warning in grid.Call(C_textBounds, as.graphicsAnnot(x$label), x$x, x$y, : font
width unknown for character 0x20

Warning in grid.Call(C_textBounds, as.graphicsAnnot(x$label), x$x, x$y, : font
width unknown for character 0x20

Warning in grid.Call(C_textBounds, as.graphicsAnnot(x$label), x$x, x$y, : font
width unknown for character 0x20

Warning in grid.Call(C_textBounds, as.graphicsAnnot(x$label), x$x, x$y, : font
width unknown for character 0x20
\end{verbatim}

\begin{verbatim}
Warning in grid.Call.graphics(C_text, as.graphicsAnnot(x$label), x$x, x$y, :
font width unknown for character 0x20

Warning in grid.Call.graphics(C_text, as.graphicsAnnot(x$label), x$x, x$y, :
font width unknown for character 0x20
\end{verbatim}

\begin{verbatim}
Warning in grid.Call(C_textBounds, as.graphicsAnnot(x$label), x$x, x$y, : font
width unknown for character 0x20

Warning in grid.Call(C_textBounds, as.graphicsAnnot(x$label), x$x, x$y, : font
width unknown for character 0x20
\end{verbatim}

\begin{verbatim}
Warning in grid.Call.graphics(C_text, as.graphicsAnnot(x$label), x$x, x$y, :
font width unknown for character 0x20
\end{verbatim}

  \begin{figure}[H]

  {\centering \includegraphics{rethinking2024wk1_files/figure-pdf/unnamed-chunk-18-1.pdf}

  }

  \end{figure}
\item
  Using the posterior distribution from 1, compute the posterior
  predictive distribution for the next 5 tosses of the same globe. Use
  the sampling method.

  The posterior predictive distribution is an extraction of data - a
  simulation - using the probabilities of the posterior distribution. We
  then examine that posterior predictive distribution to make
  descriptive statements about our questions of interest.

  Recall that we have a sample of 3 land and 11 water. Slide 90 shows
  how to draw probabilities from the posterior.
\end{enumerate}

\begin{Shaded}
\begin{Highlighting}[]
\NormalTok{post\_samples }\OtherTok{\textless{}{-}} \FunctionTok{rbeta}\NormalTok{(}\FloatTok{1e4}\NormalTok{, }\DecValTok{3}\SpecialCharTok{+}\DecValTok{1}\NormalTok{, }\DecValTok{11}\SpecialCharTok{+}\DecValTok{1}\NormalTok{)}

\FunctionTok{ggplot}\NormalTok{(}\FunctionTok{data.frame}\NormalTok{(}\AttributeTok{x=}\NormalTok{post\_samples), }\FunctionTok{aes}\NormalTok{(x)) }\SpecialCharTok{+} 
  \FunctionTok{geom\_vline}\NormalTok{(}\AttributeTok{xintercept=}\FunctionTok{c}\NormalTok{(}\FunctionTok{mean}\NormalTok{(post\_samples), }\FunctionTok{median}\NormalTok{(post\_samples)), }\AttributeTok{color=}\FunctionTok{c}\NormalTok{(}\StringTok{"darkgoldenrod2"}\NormalTok{, }\StringTok{"green"}\NormalTok{),}
             \AttributeTok{size=}\DecValTok{1}\NormalTok{) }\SpecialCharTok{+}
  \FunctionTok{geom\_density}\NormalTok{(}\AttributeTok{color=}\StringTok{"blue"}\NormalTok{,}
               \AttributeTok{fill=}\StringTok{"dodgerblue2"}\NormalTok{,}
               \AttributeTok{size=}\NormalTok{.}\DecValTok{8}\NormalTok{,}
               \AttributeTok{alpha=}\NormalTok{.}\DecValTok{4}\NormalTok{) }\SpecialCharTok{+}
  \FunctionTok{scale\_x\_continuous}\NormalTok{(}\AttributeTok{limits=}\FunctionTok{c}\NormalTok{(}\DecValTok{0}\NormalTok{,}\DecValTok{1}\NormalTok{),}
                     \AttributeTok{breaks=}\FunctionTok{seq}\NormalTok{(}\DecValTok{0}\NormalTok{,}\DecValTok{1}\NormalTok{,.}\DecValTok{1}\NormalTok{),}
                     \AttributeTok{labels=}\FunctionTok{percent\_format}\NormalTok{(}\AttributeTok{accuracy=}\DecValTok{1}\NormalTok{)) }\SpecialCharTok{+}
  \FunctionTok{theme}\NormalTok{(}\AttributeTok{axis.text.y=}\FunctionTok{element\_blank}\NormalTok{(),}
        \AttributeTok{axis.ticks.y=}\FunctionTok{element\_blank}\NormalTok{()) }\SpecialCharTok{+}
  \FunctionTok{labs}\NormalTok{(}\AttributeTok{title=}\StringTok{"Posterior predictive distribution"}\NormalTok{,}
       \AttributeTok{x=}\StringTok{"Probability of water"}\NormalTok{,}
       \AttributeTok{y=}\StringTok{""}\NormalTok{,}
       \AttributeTok{caption=}\StringTok{"Probability of water 3/14 = 21.4\%"}\NormalTok{)}
\end{Highlighting}
\end{Shaded}

\begin{verbatim}
Warning: Using `size` aesthetic for lines was deprecated in ggplot2 3.4.0.
i Please use `linewidth` instead.
\end{verbatim}

\begin{verbatim}
Warning in grid.Call(C_textBounds, as.graphicsAnnot(x$label), x$x, x$y, : font
width unknown for character 0x20

Warning in grid.Call(C_textBounds, as.graphicsAnnot(x$label), x$x, x$y, : font
width unknown for character 0x20

Warning in grid.Call(C_textBounds, as.graphicsAnnot(x$label), x$x, x$y, : font
width unknown for character 0x20

Warning in grid.Call(C_textBounds, as.graphicsAnnot(x$label), x$x, x$y, : font
width unknown for character 0x20

Warning in grid.Call(C_textBounds, as.graphicsAnnot(x$label), x$x, x$y, : font
width unknown for character 0x20

Warning in grid.Call(C_textBounds, as.graphicsAnnot(x$label), x$x, x$y, : font
width unknown for character 0x20

Warning in grid.Call(C_textBounds, as.graphicsAnnot(x$label), x$x, x$y, : font
width unknown for character 0x20

Warning in grid.Call(C_textBounds, as.graphicsAnnot(x$label), x$x, x$y, : font
width unknown for character 0x20

Warning in grid.Call(C_textBounds, as.graphicsAnnot(x$label), x$x, x$y, : font
width unknown for character 0x20

Warning in grid.Call(C_textBounds, as.graphicsAnnot(x$label), x$x, x$y, : font
width unknown for character 0x20

Warning in grid.Call(C_textBounds, as.graphicsAnnot(x$label), x$x, x$y, : font
width unknown for character 0x20

Warning in grid.Call(C_textBounds, as.graphicsAnnot(x$label), x$x, x$y, : font
width unknown for character 0x20

Warning in grid.Call(C_textBounds, as.graphicsAnnot(x$label), x$x, x$y, : font
width unknown for character 0x20

Warning in grid.Call(C_textBounds, as.graphicsAnnot(x$label), x$x, x$y, : font
width unknown for character 0x20

Warning in grid.Call(C_textBounds, as.graphicsAnnot(x$label), x$x, x$y, : font
width unknown for character 0x20

Warning in grid.Call(C_textBounds, as.graphicsAnnot(x$label), x$x, x$y, : font
width unknown for character 0x20

Warning in grid.Call(C_textBounds, as.graphicsAnnot(x$label), x$x, x$y, : font
width unknown for character 0x20

Warning in grid.Call(C_textBounds, as.graphicsAnnot(x$label), x$x, x$y, : font
width unknown for character 0x20

Warning in grid.Call(C_textBounds, as.graphicsAnnot(x$label), x$x, x$y, : font
width unknown for character 0x20

Warning in grid.Call(C_textBounds, as.graphicsAnnot(x$label), x$x, x$y, : font
width unknown for character 0x20

Warning in grid.Call(C_textBounds, as.graphicsAnnot(x$label), x$x, x$y, : font
width unknown for character 0x20

Warning in grid.Call(C_textBounds, as.graphicsAnnot(x$label), x$x, x$y, : font
width unknown for character 0x20

Warning in grid.Call(C_textBounds, as.graphicsAnnot(x$label), x$x, x$y, : font
width unknown for character 0x20

Warning in grid.Call(C_textBounds, as.graphicsAnnot(x$label), x$x, x$y, : font
width unknown for character 0x20

Warning in grid.Call(C_textBounds, as.graphicsAnnot(x$label), x$x, x$y, : font
width unknown for character 0x20

Warning in grid.Call(C_textBounds, as.graphicsAnnot(x$label), x$x, x$y, : font
width unknown for character 0x20

Warning in grid.Call(C_textBounds, as.graphicsAnnot(x$label), x$x, x$y, : font
width unknown for character 0x20

Warning in grid.Call(C_textBounds, as.graphicsAnnot(x$label), x$x, x$y, : font
width unknown for character 0x20

Warning in grid.Call(C_textBounds, as.graphicsAnnot(x$label), x$x, x$y, : font
width unknown for character 0x20

Warning in grid.Call(C_textBounds, as.graphicsAnnot(x$label), x$x, x$y, : font
width unknown for character 0x20

Warning in grid.Call(C_textBounds, as.graphicsAnnot(x$label), x$x, x$y, : font
width unknown for character 0x20

Warning in grid.Call(C_textBounds, as.graphicsAnnot(x$label), x$x, x$y, : font
width unknown for character 0x20

Warning in grid.Call(C_textBounds, as.graphicsAnnot(x$label), x$x, x$y, : font
width unknown for character 0x20

Warning in grid.Call(C_textBounds, as.graphicsAnnot(x$label), x$x, x$y, : font
width unknown for character 0x20

Warning in grid.Call(C_textBounds, as.graphicsAnnot(x$label), x$x, x$y, : font
width unknown for character 0x20

Warning in grid.Call(C_textBounds, as.graphicsAnnot(x$label), x$x, x$y, : font
width unknown for character 0x20

Warning in grid.Call(C_textBounds, as.graphicsAnnot(x$label), x$x, x$y, : font
width unknown for character 0x20

Warning in grid.Call(C_textBounds, as.graphicsAnnot(x$label), x$x, x$y, : font
width unknown for character 0x20

Warning in grid.Call(C_textBounds, as.graphicsAnnot(x$label), x$x, x$y, : font
width unknown for character 0x20

Warning in grid.Call(C_textBounds, as.graphicsAnnot(x$label), x$x, x$y, : font
width unknown for character 0x20

Warning in grid.Call(C_textBounds, as.graphicsAnnot(x$label), x$x, x$y, : font
width unknown for character 0x20

Warning in grid.Call(C_textBounds, as.graphicsAnnot(x$label), x$x, x$y, : font
width unknown for character 0x20

Warning in grid.Call(C_textBounds, as.graphicsAnnot(x$label), x$x, x$y, : font
width unknown for character 0x20

Warning in grid.Call(C_textBounds, as.graphicsAnnot(x$label), x$x, x$y, : font
width unknown for character 0x20

Warning in grid.Call(C_textBounds, as.graphicsAnnot(x$label), x$x, x$y, : font
width unknown for character 0x20

Warning in grid.Call(C_textBounds, as.graphicsAnnot(x$label), x$x, x$y, : font
width unknown for character 0x20

Warning in grid.Call(C_textBounds, as.graphicsAnnot(x$label), x$x, x$y, : font
width unknown for character 0x20

Warning in grid.Call(C_textBounds, as.graphicsAnnot(x$label), x$x, x$y, : font
width unknown for character 0x20

Warning in grid.Call(C_textBounds, as.graphicsAnnot(x$label), x$x, x$y, : font
width unknown for character 0x20
\end{verbatim}

\begin{verbatim}
Warning in grid.Call.graphics(C_text, as.graphicsAnnot(x$label), x$x, x$y, :
font width unknown for character 0x20

Warning in grid.Call.graphics(C_text, as.graphicsAnnot(x$label), x$x, x$y, :
font width unknown for character 0x20
\end{verbatim}

\begin{verbatim}
Warning in grid.Call(C_textBounds, as.graphicsAnnot(x$label), x$x, x$y, : font
width unknown for character 0x20

Warning in grid.Call(C_textBounds, as.graphicsAnnot(x$label), x$x, x$y, : font
width unknown for character 0x20

Warning in grid.Call(C_textBounds, as.graphicsAnnot(x$label), x$x, x$y, : font
width unknown for character 0x20

Warning in grid.Call(C_textBounds, as.graphicsAnnot(x$label), x$x, x$y, : font
width unknown for character 0x20
\end{verbatim}

\begin{verbatim}
Warning in grid.Call.graphics(C_text, as.graphicsAnnot(x$label), x$x, x$y, :
font width unknown for character 0x20

Warning in grid.Call.graphics(C_text, as.graphicsAnnot(x$label), x$x, x$y, :
font width unknown for character 0x20
\end{verbatim}

\begin{verbatim}
Warning in grid.Call(C_textBounds, as.graphicsAnnot(x$label), x$x, x$y, : font
width unknown for character 0x20

Warning in grid.Call(C_textBounds, as.graphicsAnnot(x$label), x$x, x$y, : font
width unknown for character 0x20

Warning in grid.Call(C_textBounds, as.graphicsAnnot(x$label), x$x, x$y, : font
width unknown for character 0x20

Warning in grid.Call(C_textBounds, as.graphicsAnnot(x$label), x$x, x$y, : font
width unknown for character 0x20

Warning in grid.Call(C_textBounds, as.graphicsAnnot(x$label), x$x, x$y, : font
width unknown for character 0x20

Warning in grid.Call(C_textBounds, as.graphicsAnnot(x$label), x$x, x$y, : font
width unknown for character 0x20

Warning in grid.Call(C_textBounds, as.graphicsAnnot(x$label), x$x, x$y, : font
width unknown for character 0x20

Warning in grid.Call(C_textBounds, as.graphicsAnnot(x$label), x$x, x$y, : font
width unknown for character 0x20

Warning in grid.Call(C_textBounds, as.graphicsAnnot(x$label), x$x, x$y, : font
width unknown for character 0x20

Warning in grid.Call(C_textBounds, as.graphicsAnnot(x$label), x$x, x$y, : font
width unknown for character 0x20
\end{verbatim}

\begin{verbatim}
Warning in grid.Call.graphics(C_text, as.graphicsAnnot(x$label), x$x, x$y, :
font width unknown for character 0x20

Warning in grid.Call.graphics(C_text, as.graphicsAnnot(x$label), x$x, x$y, :
font width unknown for character 0x20

Warning in grid.Call.graphics(C_text, as.graphicsAnnot(x$label), x$x, x$y, :
font width unknown for character 0x20

Warning in grid.Call.graphics(C_text, as.graphicsAnnot(x$label), x$x, x$y, :
font width unknown for character 0x20

Warning in grid.Call.graphics(C_text, as.graphicsAnnot(x$label), x$x, x$y, :
font width unknown for character 0x20
\end{verbatim}

\begin{figure}[H]

{\centering \includegraphics{rethinking2024wk1_files/figure-pdf/unnamed-chunk-19-1.pdf}

}

\end{figure}

Now that we have our posterior predictive distribution, we can predict
the probability of any \emph{p}.

\begin{Shaded}
\begin{Highlighting}[]
\NormalTok{W\_sim }\OtherTok{\textless{}{-}} \FunctionTok{rbinom}\NormalTok{(}\FloatTok{1e4}\NormalTok{, }\AttributeTok{size=}\DecValTok{5}\NormalTok{, }\AttributeTok{p=}\NormalTok{post\_samples)}

\FunctionTok{ggplot}\NormalTok{(}\FunctionTok{data.frame}\NormalTok{(}\AttributeTok{x=}\NormalTok{W\_sim), }\FunctionTok{aes}\NormalTok{(x)) }\SpecialCharTok{+} 
  \FunctionTok{geom\_bar}\NormalTok{(}\AttributeTok{width=}\NormalTok{.}\DecValTok{2}\NormalTok{,}
           \AttributeTok{color=}\StringTok{"blue"}\NormalTok{,}
           \AttributeTok{fill=}\StringTok{"dodgerblue2"}\NormalTok{,}
           \AttributeTok{alpha=}\NormalTok{.}\DecValTok{6}\NormalTok{) }\SpecialCharTok{+}
  \FunctionTok{scale\_x\_continuous}\NormalTok{(}\AttributeTok{breaks=}\DecValTok{0}\SpecialCharTok{:}\DecValTok{5}\NormalTok{) }\SpecialCharTok{+}
  \FunctionTok{theme}\NormalTok{(}\AttributeTok{axis.text.y=}\FunctionTok{element\_blank}\NormalTok{(),}
        \AttributeTok{axis.ticks.y=}\FunctionTok{element\_blank}\NormalTok{()) }\SpecialCharTok{+}
  \FunctionTok{labs}\NormalTok{(}\AttributeTok{x=}\StringTok{"Counts of water"}\NormalTok{,}
       \AttributeTok{y=}\StringTok{""}\NormalTok{,}
       \AttributeTok{title=}\StringTok{"Posterior predictive distribution}\SpecialCharTok{\textbackslash{}n}\StringTok{for five additional coin tosses"}\NormalTok{)}
\end{Highlighting}
\end{Shaded}

\begin{verbatim}
Warning in grid.Call(C_textBounds, as.graphicsAnnot(x$label), x$x, x$y, : font
width unknown for character 0x20

Warning in grid.Call(C_textBounds, as.graphicsAnnot(x$label), x$x, x$y, : font
width unknown for character 0x20

Warning in grid.Call(C_textBounds, as.graphicsAnnot(x$label), x$x, x$y, : font
width unknown for character 0x20

Warning in grid.Call(C_textBounds, as.graphicsAnnot(x$label), x$x, x$y, : font
width unknown for character 0x20

Warning in grid.Call(C_textBounds, as.graphicsAnnot(x$label), x$x, x$y, : font
width unknown for character 0x20

Warning in grid.Call(C_textBounds, as.graphicsAnnot(x$label), x$x, x$y, : font
width unknown for character 0x20

Warning in grid.Call(C_textBounds, as.graphicsAnnot(x$label), x$x, x$y, : font
width unknown for character 0x20

Warning in grid.Call(C_textBounds, as.graphicsAnnot(x$label), x$x, x$y, : font
width unknown for character 0x20

Warning in grid.Call(C_textBounds, as.graphicsAnnot(x$label), x$x, x$y, : font
width unknown for character 0x20

Warning in grid.Call(C_textBounds, as.graphicsAnnot(x$label), x$x, x$y, : font
width unknown for character 0x20

Warning in grid.Call(C_textBounds, as.graphicsAnnot(x$label), x$x, x$y, : font
width unknown for character 0x20

Warning in grid.Call(C_textBounds, as.graphicsAnnot(x$label), x$x, x$y, : font
width unknown for character 0x20

Warning in grid.Call(C_textBounds, as.graphicsAnnot(x$label), x$x, x$y, : font
width unknown for character 0x20

Warning in grid.Call(C_textBounds, as.graphicsAnnot(x$label), x$x, x$y, : font
width unknown for character 0x20

Warning in grid.Call(C_textBounds, as.graphicsAnnot(x$label), x$x, x$y, : font
width unknown for character 0x20

Warning in grid.Call(C_textBounds, as.graphicsAnnot(x$label), x$x, x$y, : font
width unknown for character 0x20

Warning in grid.Call(C_textBounds, as.graphicsAnnot(x$label), x$x, x$y, : font
width unknown for character 0x20

Warning in grid.Call(C_textBounds, as.graphicsAnnot(x$label), x$x, x$y, : font
width unknown for character 0x20

Warning in grid.Call(C_textBounds, as.graphicsAnnot(x$label), x$x, x$y, : font
width unknown for character 0x20

Warning in grid.Call(C_textBounds, as.graphicsAnnot(x$label), x$x, x$y, : font
width unknown for character 0x20

Warning in grid.Call(C_textBounds, as.graphicsAnnot(x$label), x$x, x$y, : font
width unknown for character 0x20

Warning in grid.Call(C_textBounds, as.graphicsAnnot(x$label), x$x, x$y, : font
width unknown for character 0x20

Warning in grid.Call(C_textBounds, as.graphicsAnnot(x$label), x$x, x$y, : font
width unknown for character 0x20

Warning in grid.Call(C_textBounds, as.graphicsAnnot(x$label), x$x, x$y, : font
width unknown for character 0x20

Warning in grid.Call(C_textBounds, as.graphicsAnnot(x$label), x$x, x$y, : font
width unknown for character 0x20

Warning in grid.Call(C_textBounds, as.graphicsAnnot(x$label), x$x, x$y, : font
width unknown for character 0x20

Warning in grid.Call(C_textBounds, as.graphicsAnnot(x$label), x$x, x$y, : font
width unknown for character 0x20

Warning in grid.Call(C_textBounds, as.graphicsAnnot(x$label), x$x, x$y, : font
width unknown for character 0x20

Warning in grid.Call(C_textBounds, as.graphicsAnnot(x$label), x$x, x$y, : font
width unknown for character 0x20

Warning in grid.Call(C_textBounds, as.graphicsAnnot(x$label), x$x, x$y, : font
width unknown for character 0x20

Warning in grid.Call(C_textBounds, as.graphicsAnnot(x$label), x$x, x$y, : font
width unknown for character 0x20

Warning in grid.Call(C_textBounds, as.graphicsAnnot(x$label), x$x, x$y, : font
width unknown for character 0x20

Warning in grid.Call(C_textBounds, as.graphicsAnnot(x$label), x$x, x$y, : font
width unknown for character 0x20

Warning in grid.Call(C_textBounds, as.graphicsAnnot(x$label), x$x, x$y, : font
width unknown for character 0x20

Warning in grid.Call(C_textBounds, as.graphicsAnnot(x$label), x$x, x$y, : font
width unknown for character 0x20

Warning in grid.Call(C_textBounds, as.graphicsAnnot(x$label), x$x, x$y, : font
width unknown for character 0x20

Warning in grid.Call(C_textBounds, as.graphicsAnnot(x$label), x$x, x$y, : font
width unknown for character 0x20

Warning in grid.Call(C_textBounds, as.graphicsAnnot(x$label), x$x, x$y, : font
width unknown for character 0x20

Warning in grid.Call(C_textBounds, as.graphicsAnnot(x$label), x$x, x$y, : font
width unknown for character 0x20

Warning in grid.Call(C_textBounds, as.graphicsAnnot(x$label), x$x, x$y, : font
width unknown for character 0x20

Warning in grid.Call(C_textBounds, as.graphicsAnnot(x$label), x$x, x$y, : font
width unknown for character 0x20

Warning in grid.Call(C_textBounds, as.graphicsAnnot(x$label), x$x, x$y, : font
width unknown for character 0x20

Warning in grid.Call(C_textBounds, as.graphicsAnnot(x$label), x$x, x$y, : font
width unknown for character 0x20

Warning in grid.Call(C_textBounds, as.graphicsAnnot(x$label), x$x, x$y, : font
width unknown for character 0x20
\end{verbatim}

\begin{verbatim}
Warning in grid.Call.graphics(C_text, as.graphicsAnnot(x$label), x$x, x$y, :
font width unknown for character 0x20

Warning in grid.Call.graphics(C_text, as.graphicsAnnot(x$label), x$x, x$y, :
font width unknown for character 0x20
\end{verbatim}

\begin{verbatim}
Warning in grid.Call(C_textBounds, as.graphicsAnnot(x$label), x$x, x$y, : font
width unknown for character 0x20

Warning in grid.Call(C_textBounds, as.graphicsAnnot(x$label), x$x, x$y, : font
width unknown for character 0x20

Warning in grid.Call(C_textBounds, as.graphicsAnnot(x$label), x$x, x$y, : font
width unknown for character 0x20

Warning in grid.Call(C_textBounds, as.graphicsAnnot(x$label), x$x, x$y, : font
width unknown for character 0x20

Warning in grid.Call(C_textBounds, as.graphicsAnnot(x$label), x$x, x$y, : font
width unknown for character 0x20

Warning in grid.Call(C_textBounds, as.graphicsAnnot(x$label), x$x, x$y, : font
width unknown for character 0x20

Warning in grid.Call(C_textBounds, as.graphicsAnnot(x$label), x$x, x$y, : font
width unknown for character 0x20

Warning in grid.Call(C_textBounds, as.graphicsAnnot(x$label), x$x, x$y, : font
width unknown for character 0x20

Warning in grid.Call(C_textBounds, as.graphicsAnnot(x$label), x$x, x$y, : font
width unknown for character 0x20

Warning in grid.Call(C_textBounds, as.graphicsAnnot(x$label), x$x, x$y, : font
width unknown for character 0x20

Warning in grid.Call(C_textBounds, as.graphicsAnnot(x$label), x$x, x$y, : font
width unknown for character 0x20

Warning in grid.Call(C_textBounds, as.graphicsAnnot(x$label), x$x, x$y, : font
width unknown for character 0x20
\end{verbatim}

\begin{verbatim}
Warning in grid.Call.graphics(C_text, as.graphicsAnnot(x$label), x$x, x$y, :
font width unknown for character 0x20

Warning in grid.Call.graphics(C_text, as.graphicsAnnot(x$label), x$x, x$y, :
font width unknown for character 0x20

Warning in grid.Call.graphics(C_text, as.graphicsAnnot(x$label), x$x, x$y, :
font width unknown for character 0x20

Warning in grid.Call.graphics(C_text, as.graphicsAnnot(x$label), x$x, x$y, :
font width unknown for character 0x20

Warning in grid.Call.graphics(C_text, as.graphicsAnnot(x$label), x$x, x$y, :
font width unknown for character 0x20

Warning in grid.Call.graphics(C_text, as.graphicsAnnot(x$label), x$x, x$y, :
font width unknown for character 0x20
\end{verbatim}

\begin{figure}[H]

{\centering \includegraphics{rethinking2024wk1_files/figure-pdf/unnamed-chunk-20-1.pdf}

}

\end{figure}

Consistent with an expected probability of 11 percent, the most likely
outcome of five tosses is 1 water. Zero outomes of water is the
second-most likely outcome of five tosses.



\end{document}
